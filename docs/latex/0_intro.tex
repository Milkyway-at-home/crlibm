


\section{Correct rounding and elementary functions}
\label{sect:intro}

The need for accurate elementary functions is important in many
critical programs.  Methods for computing these functions include
table-based methods\cite{Far81,Tan91}, polynomial approximations and
mixed methods\cite{DauMor2k}. See the books by Muller\cite{Muller97} or
Markstein\cite{Markstein2000} for recent surveys on the subject.

The IEEE-754 standard for floating-point arithmetic\cite{IEEE754}
defines the usual floating-point formats (single and double
precision). It also specifies the behavior of the four basic operators
($+,-,\times,\div$) and the square root in four rounding modes (to the
nearest, towards $+\infty$, towards $-\infty$ and towards $0$). Its
adoption and widespread use have increased the numerical quality of,
and confidence in floating-point code. In particular, it has improved
\emph{portability} of such code and allowed construction of
\emph{proofs} on its numerical behavior. Directed rounding modes
(towards $+\infty$, $-\infty$ and $0$) also enabled efficient
\emph{interval arithmetic}\cite{Moore66,KKLRW93}.

However, the IEEE-754 standard specifies nothing about elementary
functions, which limits these advances to code excluding such
functions.  Currently, several options exist: on one hand, one can use
today's mathematical libraries that are efficient but without any
warranty on the correctness of the results. When strict guarantees are
needed, some multiple-precision packages like MPFR \cite{MPFRweb}
offer correct rounding in all rounding modes, but are several orders
of magnitude slower than the usual mathematical libraries for the same
precision. The recently released IBM Ultimate Math
Library\cite{IBMlibultimweb} is both portable and fast, if bulky. It
claims to offer correct rounding to the nearest, however this claim is
not proven, and indeed some values are still uncorrectly
rounded\footnote{Try
  $\sin(0.252113814338773334355892075109295547008514404296875)$}.
Besides, this library still lacks directed rounding modes needed for
interval arithmetic.


The  goal of the \crlibm\ project is to build on a combination of several
recent advances to design a proven  correctly rounded mathematical
library which is fast enough to replace the existing libraries, at acceptable
 cost in terms of performance and resources.




%%%%%%%%%%%%%%%%%%%%%%%%%%%%%%%%%%%%%%%%%%%%%%%%%%%%%%%%%%%%%
\section{A methodology for efficient correctly-rounded functions}
\label{section:methodology}


\subsection{The Table Maker's Dilemma}

With a few exceptions, the image $\hat{y}$ of a floating-point number $x$ by
a transcendental function $f$ is a transcendental number, and can
therefore not be represented exactly in standard numeration systems.
The only hope is to compute the floating-point number that is closest
to (resp.  immediately above or immediately below) the mathematical
value, which we call the result \emph{correctly rounded} to the
nearest (resp.  towards $+\infty$ or towards $-\infty$).

It is only possible to compute an approximation ${y}$ to the real
number $\hat{y}$ with precision $\maxeps{}$. This ensures that the real value
$\hat{y}$ belongs to the interval $[{y}(1-\maxeps{}) , {y}(1+\maxeps{})]$.
Sometimes however, this information is not enough to decide correct
rounding. For example, if $[{y}(1-\maxeps{}) , {y}(1+\maxeps{})]$
contains the middle of two consecutive floating-point numbers, it is
impossible to decide which of these two numbers is the correctly
rounded to the nearest of $\hat{y}$. This is known as the Table Maker's
Dilemma (TMD).

\subsection{The onion peeling strategy}

A method described by Ziv \cite{Ziv91} is to increase the precision
$\maxeps$ of the approximation until the correctly rounded value can
be decided.  Given a function $f$ and an argument $x$, the value of
$f(x)$ is first evaluated using a quick approximation of precision
$\maxeps_1$.  Knowing $\maxeps_1$, it is possible to decide if
rounding is possible, or if more precision is required, in which case
the computation is restarted using a slower approximation of precision
$\maxeps_2$ greater than $\maxeps_1$, and so on. This approach makes
sense even in terms of average performance, as the slower steps are
rarely taken.

However there was until recently no practical bound on the termination
time of such an algorithm. This iteration has been proven to
terminate, but the actual maximal precision required in the worst case
is unknown.  This might prevent using this method in critical
application.




\section{The Correctly Rounded Mathematical Library}
\label{section:crlibm}

Our own library, called \crlibm\ for \emph{correctly rounded
  mathematical library}, is based on the work of
Lef\`evre and Muller \cite{LMT98,Lef2000} who computed the worst-case $\maxeps$
required for correctly rounding several functions in double-precision
over selected intervals in the four IEEE-754 rounding modes. For
example, they proved that 157 bits are enough to ensure correct rounding
of the exponential function on all of its domain for the four IEEE-754
rounding modes.

\subsection{Two steps are enough}
Thanks to such results, we are able to guarantee correct rounding in
two iterations only, which we may then optimize separately. The first
of these iterations is relatively fast and provides between 60 and 80
bits of accuracy (depending on the function), which is sufficient in
most cases. It will be referred throughout this document as the \quick\ 
phase of the algorithm. The second phase, referred to as the
\accurate\ phase, is dedicated to challenging cases. It is slower but
has a reasonably bounded execution time, tightly targeted at
Lef\`evre's worst cases.

Having a proven worst-case execution time lifts the last obstacle to a
generalization of correctly rounded transcendentals. Besides, having
only two steps allows us to publish, along with each function, a proof
of its correctly rounding behavior.


\subsection{Portable IEEE-754 FP for fast first step}
The computation of a tight bound on the approximation error of the
first step ($\maxeps_1$) is crucial for the efficiency of the onion
peeling strategy: overestimating $\maxeps_1$ means going more often
than needed through the second step, as will be detailed below in
\ref{sec:error-accuracy-perf}. As we want the proof to be portable as
well as the code, our first steps are written in strict IEEE-754
arithmetic. On some systems, this means preventing the
compiler/processor combination to use advanced floating-point features
such as fused multiply-and-add or extended double precision. It also
means that the performance of our portable library will be lower than
optimized libraries using these features (see \cite{DinGast2005} for
recent research on processor-specific correct-rounding).

To ease these proofs, our first steps make wide use of classical, well
proven results like Sterbenz' lemma or other floating-point theorems.
When a result is needed in a precision higher than double precision
(as is the case of $y_1$, the result of the first step), it is
represented as as the sum of two floating-point numbers, also called a
\emph{double-double} number.  There are well-known algorithms for
computing on double-doubles, and they are presented in the next
chapter. An advantage of properly encapsulating double-double
arithmetic is that we can actually exploit fused multiply-and-add
operators in a transparent manner (this experimental feature is
currently available for the Itanium and PowerPC platforms, when using
the \texttt{gcc} compiler).

At the end of the \quick\ phase, a sequence of simple tests on
$y_1$ knowing $\maxeps_1$ allows to decide whether to go for
the second step. The sequence corresponding to each rounding mode is
shared by most functions and is also carefully proven in the next
chapter.


\subsection{Software Carry-Save for an accurate second step}
For the second step, we designed an ad-hoc multiple-precision library
called Software Carry-Save library \emph{(scslib)} which is lighter
and faster than other available libraries for this specific
application \cite{DefDin2002,DinDef2003}. This choice is motivated by
considerations of code size and performance, but also by the need to
be independent of other libraries: Again, we need a library on which
we may rely at the proof level. This library is included in \crlibm,
but also distributed separately \cite{SCSweb}.




\subsection{Proving the correct rounding property}

Throughout this document, what we call ``proving'' a function mostly
means proving a tight bound on the total relative error of our
evaluation scheme. The actual proof of the correct rounding property
is then dependent on the availability of an actual worst-case accuracy
for correctly rounding this function, as computed by Lef\`evre and
Muller. Three cases may happen:
\begin{itemize}
\item The worst case have been computed over the whole domain of the
  function. In this case the correct rounding property for this
  function is fully proven. As of \texttt{crlibm0.8}, this is the case
  of the exponential and natural logarithm functions, and for the
  hyperbolic sine and cosine.

\item The worst cases have been computed only over a subdomain of the
  function. Then the correct rounding property is proven on this
  subdomain. Outside of this domain \texttt{crlibm} offers
  ``astronomical confidence'' that the function is correctly rounded:
  to the best of current knowledge \cite{Gal86, DinGast2005}, the
  probability of the existence of a misrounded value in the function's
  domain is lower than $2^{-40}$. As of \texttt{crlibm0.8}, this is
  the case of the trigonometric functions and the arctangent. The
  actual domain on which the proof is complete is mentionned in the
  respective chapter of this document, and summed up below in \ref{sec:current-state}.
\item The search for worst cases hasn't begun yet.
\end{itemize}

We acknowledge that the notion of astronomical confidence breaks the
objective of guaranteed correct rounding, and we sidestep this problem
by publishing along with the library (in this document) the domain of
full confidence, which will only expand as time goes.  Such behaviour
has been proposed as a standard in \cite{DefHanLefMulRevZim2004}.  The
main advantage of this approach is that it ensures bounded and
consistent worst-case execution time (within a factor 100 of that of
the best available faithful \texttt{libm}s), which we believe is
crucial to the generalization of correctly rounded functions.

The alternative to our approach would be to implement a
multi-layer onion-peeling strategy, as do GNU MPFR and Sun's
\texttt{libmcr}. There are however many drawbacks to this approach, too:

\begin{itemize}

\item One may argue that, due to the finite nature of computers, it
  only pushes the bounds of astronomy a little bit further.

\item The multilayer approach is only proven to terminate on
  elementary functions: the termination proof needs a theorem stating
  for example that the image of a rational by the function (with some
  well-known exceptions) will not be a rational. For other library
  functions like  special functions, we have no such theorem.
  For these functions, we prefer take the risk of a misrounded
  value than the risk of an infinite loop.

\item Similarly, the multilayer approach has potentially unbounded
  execution time and memory consumption which make it unsuitable for
  real-time or safety-critical applications, whereas crlibm will only
  be unsuitable if the safety depends on correct rounding, which is
  much less likely.

\item Multilayer code is probably much more complex and error prone.
  One important problem is that it contains code that, according all
  probabilities, will never be run. Therefore, testing this code can
  not be done on the final production executable, but on a different
  executable in which previous layers have been disabled. This
  introduces the possibility of undetected bugs in the final
  production executable.

\end{itemize}

In the future, we will add, to those \texttt{crlibm} functions for
which the required worst-case accuracy is unknown, a misround
detection test at the end of the second step. This test will either
print out on standard error a lengthy warning inviting to report this
case, or launch MPFR computation, depending on a compilation switch.
% TODO



\subsection{Error analysis and the accuracy/performance tradeoff
  \label{sec:error-accuracy-perf}}

As there are two step on the evaluation, the proof also usually
consists of two parts. The code of the second, accurate step is
usually very simple and straightforward:
\begin{itemize}
\item Performance is not that much of an issue, since this step is rarely taken.
\item All the special cases have already been filtered by the first step.
\item The \texttt{scslib} library provides an overkill of precision.
\end{itemize}

Therefore, the error analysis of the second step, which ultimately
proves the correct rounding property, is not very difficult.

For the first step, however, things are more complicated:
\begin{itemize}
\item We have to handle special cases (infinities, NaNs, signed
  zeroes, over- and underflows).
\item Performance is a primary concern, sometimes leading to ``dirty
  tricks'' obfuscating the code.
\item We have to compute a \emph{tight} error bound, as explained below.
\end{itemize}

Why do we need a tight error bound? Because the decision to launch the
second step is taken by a \emph{rounding test}  depending on
\begin{itemize}
\item the approximation $y_h+y_l$ computed in the first step, and
\item this error bound $\maxeps_1$, which is computed statically.
\end{itemize}

The various rounding tests are detailed and proven in
\ref{section:testrounding}.  The important notion here is that
\emph{the probability of launching the second, slower step will be
  proportional to the error bound $\maxeps_1$ computed for the first step}.

This defines the main performance tradeoff one has to manage when
designing a correctly-rounded function: The average evaluation time
will be
\begin{equation}
  T_{\mbox{\small avg}} = T_1 + p_2T_2 \label{eq:Tavg}
\end{equation}
where $T_1$ and $T_2$ are the execution time of the first and second
phase respectively (with $T_2\approx 100T_1$ in \crlibm), and $p_2$ is
the probability of launching the second phase (typically we aim at
$p_2=1/1000$ so that the average cost of the second step is less than
$10\%$ of the total.  

As $p_2$ is almost proportional to $\maxeps_1$, to minimise the average
time, we have to
\begin{itemize}
\item balance $T_1$ and $p_2$: this is a performance/precision
  tradeoff (the faster the first step, the less accurate)
\item and compute a tight bound on the overall error  $\maxeps_1$.
\end{itemize}

Computing this tight bound is the most time-consuming part in the
design of a correctly-rounded elementary function. The proof of the
correct rounding property only needs a proven bound, but a loose bound
will mean a larger $p_2$ than strictly required, which directly
impacts average performance. Compare $p_2=1/1000$ and $p_2=1/500$ for
$T_2=100T_1$, for instance. As a consequence, when there are multiple
computation paths in the algorithm, it makes sense to precompute
different values of $\maxeps_1$ on these different paths (see for
instance the arctangent and the logarithm).






Apart from these considerations, computing the errors is mostly
textbook science. Care must be taken that only \emph{absolute} error terms
(noted $\delta$) can be added, although some error terms are best
expressed as \emph{relative} (noted $\epsilon$), like the the rounding
error of an IEEE operation, or the minimax approximation error. Remark
also that the error needed for the theorems in
\ref{section:testrounding} is a \emph{relative} error.  Managing the
relative and absolute error terms is very dependent on the function,
and usually involves keeping upper and lower bounds on the values
manipulated along with the error terms.

Error terms to consider are the following:
\begin{itemize}
\item approximation errors  (minimax or Taylor),
\item rounding error, which fall into two categories:
  \begin{itemize}
  \item roundoff errors in values tabulated as doubles or
    double-doubles (with the exception of roundoff errors on the coefficient
    of a polynomial, which are counted in the appproximation error),
  \item roundoff errors in IEEE-compliant operations.
  \end{itemize}
\end{itemize}



\subsection{Current state of \emph{crlibm} 
\label{sec:current-state} }

The \texttt{crlibm} library currently offers accurate parts for the exponential,
logarithm in radix $2$, $10$ and $e$, sine, cosine, tangent,
arctangent, plus trigonometric argument reduction. The quick part and
its proof have been written for the exponential, natural logarithm,
sine, cosine tangent, arctangent, and the hyperbolic sine and cosine
thus far.

The correct rounding property is fully proven (the worst-case required
accuracy has been determined) on the following intervals ($\mathbb{F}$
denotes the set of IEEE-754 doubles, and indicate a fully proven
function):

\begin{center}
  \begin{tabular}{|c|c|}
    \hline
    Function & interval\\
    \hline
    \hline
    $\exp$ & $\mathbb{F}$  \\
    \hline
    $\log$ & $\mathbb{F}$  \\
    \hline
    $\cosh$ & $\mathbb{F}$  \\
    \hline
    $\sinh$ & $\mathbb{F}$  \\
    \hline
    $\sin$  & $[-\pi/2, \pi/2]$  \\
    % \hline
    % $\arcsin$  & $[-\sin(2), \sin(2)]$  \\
    \hline
    $\cos$  & $[-2, 2]$  \\
    \hline
    $\tan$  & $[-2^{-3}, 2^{-3}]$  \\
    \hline
    $\arctan$  & $[-\tan(2^{-3}), \tan(2^{-3})]$ \\
    \hline
  \end{tabular}

\end{center}

\section{An overview of other  available mathematical libraries\label{section:lib-overview}}

Many high-quality mathematical libraries are freely available and have
been a source of inspiration for this work.

Most mathematical libraries do not offer correct rounding. They can be classified as 
\begin{itemize}
\item portable libraries  assuming IEEE-754
  arithmetic, like \emph{fdlibm}, written by Sun\cite{FDLIBMweb};
\item  Processor-specific libraries, by
  Intel\cite{HarKubStoTan99,IntelOpenSource} and
  HP\cite{Markstein2000,Markstein2001} among other.
\end{itemize}

Operating systems often include several mathematical libraries, some of which are derivatives of one
of the previous.

Three libraries offer correct correct rounding:
\begin{itemize}
\item The \emph{libultim} library, also called MathLib, is developed at
  IBM by Ziv and others \cite{IBMlibultimweb}. It provides correct rounding,
  under the assumption that 800 bits are enough in all case. This
  approach suffers two weaknesses. The first is the absence of proof
  that 800 bits are enough: all there is is a very high probability.
  The second is that, as we will see in the sequel, for challenging
  cases, 800 bits are much of an overkill, which can increase the
  execution time up to 20,000 times a normal execution. This will
  prevent such a library from being used in real-time applications.
  Besides, to prevent this worst case from degrading average
  performance, there is usually some intermediate levels of precision
  in MathLib's elementary functions, which makes the code larger, more
  complex, and more difficult to prove.
  
  In addition this library provides correct rounding only to nearest.
  This is the most used rounding mode, but it might not be the most
  important as far as correct rounding is concerned: correct rounding
  provides a precision improvement over current mathematical libraries
  of only a fraction of a {unit in the last place} \emph{(ulp)}.
  Conversely, the three other rounding modes are needed to guarantee
  intervals in interval arithmetic.  Without correct rounding in these
  directed rounding modes, interval arithmetic looses up to one
  \emph{ulp} of precision in each computation.
  
\item \emph{MPFR} is a multiprecision package safer than
  \emph{libultilm} as it uses arbitrary multiprecision. It provides
  most of elementary functions for the four rounding modes defined by
  the IEEE-754 standard. However this library is not optimized for
  double precision arithmetic. In addition, as its exponent range is
  much wider than that of IEEE-754, the subtleties of denormal numbers
  are difficult to handle properly using such a multiprecision
  package.

\item The \texttt{libmcr} library, by K.C. Ng, Neil Toda and others at
  Sun Microsystems, had its first beta version published in december
  2004. Its purpose is to be a reference implementation for correctly
  rounded functions in double precision. It has very clean code,
  offers arbitrary multiple precision unlike \texttt{libultim}, at the
  expense of slow performance (due to, for example dynamic allocation
  of memory). It offers the directed rounding modes, and rounds in the
  mode read from the processor status flag.
\end{itemize}



\section{Organization of the source code}

For each function, the file containing the source code for the
accurate phase is named after the function itself (for instance
\texttt{exp.c}, \texttt{log.c}), and the quick phase, when available,
is named with the \texttt{\_fast} suffix (for instance
\texttt{exp\_fast.c}). The names of auxiliary files \texttt{.c} or
\texttt{.h} files relative to a function are also prefixed with the
name of the function.

The accurate phase relies on \texttt{scslib}, the \emph{software
  carry-save} multiple-precision library written for this purpose.
This library is contained in a subdirectory called \texttt{scs\_lib}.

The common C routines that are detailed in Chapter~\ref{chap:common} of
this document are defined in \texttt{crlibm\_private.c} and
\texttt{crlibm\_private.h}.

Many of the constants used in the C code have been computed thanks to
Maple procedures which are contained in the \texttt{maple}
subdirectory. Some of these procedures are explained in
Chapter~\ref{chap:common}. For some functions, a Maple procedure
mimicking the C code, and used for debugging or optimization purpose,
is also available.


The code also includes programs to test the \texttt{crlibm} functions
against MPFR and \texttt{libultim}, in terms of correctness and
performance. They are located in the \texttt{tests} directory.


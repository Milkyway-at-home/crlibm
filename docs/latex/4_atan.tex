This chapter is contributed by Nicolas Gast under the supervision of
F. de~Dinechin.  

\newcommand{\xred}{X_{\mathrm{red}}}
\newcommand{\xredhi}{X_{\mathrm{red hi}}}
\newcommand{\xredlo}{X_{\mathrm{red lo}}}

\section{Overview}

We compute atan in two steps : the first one gives us a precision around 64
bits. The second one computes 136 bits of precision is order to have
correct rounding in all cases.

\subsubsection{Definition interval and exceptional cases}

The inverse tangent is defined over all real numbers.

\begin{itemize}
\item If $x = NaN$ , then $\arctan(x)$ should return $NaN$
\item If $x = \pm\infty$ , then $\arctan(x)$ should return
$\pm\round(\pi/2) = \pm\rounddown(\pi/2)$ in rounding to nearest mode. In
directed rounding modes, we may return $\pm\roundup(\pi/2)$ which
invalidates the inequation $\arctan(\frac{\pi}{2})$ but respects the
rounding.
\end{itemize}
we choose to return $\pm\round(\pi/2)$ when $|x|>2^{54}$ since
$\pm\round(\pi/2) =\pm\rounddown(\pi/2) $. That could be a probem for
rounding up but we choose not to have a x with $\arctan(x) > \pi/2$.
\section{Quick phase}

We have five function. One (\texttt{atan\_quick}) which returns two doubles
($atanhi$ and $atanlo$) that represent $\arctan(x)$ with a precision of
about 64 bits. Four other functions compute the correct rounding :
\texttt{atan\_rn}, \texttt{atan\_ru}, \texttt{atan\_rd} and
\texttt{atan\_rz}. First we'll see the main algorithm and then the rounding.

\subsection{Overview of the algorithm.}

We try to have about 64 bits of precision. This phase is computed in double
or double-double.

There are two steps in the algorithm: an argument reduction and a polynomial
approximation with a polynomial of degree 9. 

We compute $\arctan(x)$ as 
\begin{equation}
\arctan(x) = \arctan( b_i ) + \arctan(\frac{x-b_i}{1+x.b_i}) \label{eq:arctan_redu}
\end{equation}

The $b_i$ are exact double and $\arctan(b_i)$ are stored in
double-double.

We defined $\xred = \dfrac{x-b_i}{1+x.b_i}$ for the rest of this chapter.

We build 62 intervals $[a_i;a_{i+1}]$ and 62 $b_i$ in order that $ x \in
[a_i;a_{i+1}] \Rightarrow \dfrac{x-b_i}{1+x.b_i} < e$

We make a dichotomy in order to find $i$ such as $ x \in [a_i;a_{i+1}]
$. That's why we choose 62 $b_i$ and $e=2^{-6.3}$ (since 62 is close to
$2^6$ and a power of 2 is better for dichotomy).

Then we use a polynomial of degree 9 for the approximation of $\arctan(\xred)$:
in order to have 66 bits of precision.

\begin{equation}
\begin{split} \arctan(x)& \approx x - \dfrac{1}{3} .x^3 + \frac{1}{5}.x^5
- \frac{1}{7}.x^7 + \frac{1}{9}.x^9 \nonumber \\ \label{eq:poly_eval}
  & \approx x . + x.Q(x^2)
\end{split}
\end{equation}
 
Q is evaluated thanks to a Horner scheme:
$ Q(z) = z. (-\frac{1}{3} + z.(\frac{1}{5} + z.(-\frac{1}{7} +
z.\frac{1}{9}))) $
where each operation is computed in double.

As $|z| \leq e$, $Q(z) \leq e^2$

At the end, the reconstruction implements equation and (\ref{eq:poly_eval})  
(\ref{eq:arctan_redu}) in double-double
arithmetic.

%\begin{equation} \arctan(x) = \arctan( b_i ) + \arctan(\frac{x-b_i}{1+x.b_i}).\label{eq:\arctan_redu}
%\end{equation}

\bigskip
\subsection{Error analysis on atan\_quick}

We choose four rounding constant : two when there is a argument reduction, two in
the other case. For each case, we made two constants on order to
improve performances. We will compute the error separately.

An accurate computation is done in \texttt{maple/arctan\_coef.mw} 

\paragraph{Notes on $b_i$, $a_i$ and $\arctan(b_i)$}
The $b_i$ and $a_i$ are computed thanks to the \texttt{allbi} maple
procedure (see \texttt{maple/arctan\_coef.mw}). There is no approximation
error on $b_i$ since they are computed (in the maple procedure) as
double. $\arctan (b_i)$ are stored in double-double so there is an
approximation of $2^{-105}$ on them. We have a possible error about $a_i$
because they are real number stored as double but we have a margin on $a_i$
that prove us that $\xred$ is always smaller than $e$.

\subsubsection{Error about argument reduction}

\begin{lstlisting}[caption={Reduction part 1},firstnumber=1]
  if (x > MIN_REDUCTION_NEEDED) /* test if reduction is necessary : */ 
  {
    double xmBIhi,xmBIlo;      

      if (x > arctan_table[61][B].d) {
        i=61;
        Add12( xmBihi , xmBilo , x , -arctan_table[61][B].d);
      }
      else
        {
          /* compute i so that a[i] < x < a[i+1] */
          i=31;
          if (x < arctan_table[i][A].d) i-= 16;
          else i+=16;
          if (x < arctan_table[i][A].d) i-= 8;
          else i+= 8;
          if (x < arctan_table[i][A].d) i-= 4;
          else i+= 4;
          if (x < arctan_table[i][A].d) i-= 2;
          else i+= 2;
          if (x < arctan_table[i][A].d) i-= 1;
          else i+= 1;
          if (x < arctan_table[i][A].d) i-= 1
          xmBihi = x-arctan_table[i][B].d;
          xmBilo = 0.0;
        }
      
\end{lstlisting}

\begin{tabular}{ll}
Lines  1 & test if $x > 2^{-6.3}$ and so need to be reduced\\
Line 5 & test if $x>b[61]$ because when $i \in [0;60] : b_i/2 < x <
b_i$ (or $ x/2 < b_i < x$) and then \\&$x-b_i$ is computed exactly
value thanks to Sterbenz lemma.\\
Line 10...21 & compute $i$ so that $\frac{x-b_i}{1+x.b_i} < 2^{-6.3} $\\
Line 7 and 23 & compute $xmBIhi + xmBIlo = x - b_i$

\end{tabular}

We have no rounding error in the computation of $x-b_i$.

\begin{lstlisting}[caption={Reduction part 2},firstnumber=1]
      Mul12(&tmphi,&tmplo, x, arctan_table[i][B].d);

      if (tmphi > 1)
        Add22(&x0hi,&x0lo,tmphi,tmplo, 1.0,0.0);
      else {Add22( &x0hi , &x0lo , 1.0,0.0,tmphi,tmplo);}
      
      DIV2( xmBihi , xmBilo , x0hi,x0lo, xhi,xlo);

\end{lstlisting}

\begin{tabular}{ll}
Line 1 & compute $x.b_i$. The error is 0 thanks to Mul12\\
Line 3-5 & We need to have a Add22Comp but as we know that $x.b_i > 0$ (so
$tmphi>0$), we test if\\& $tmphi$ is greater than 1 in order to be
faster. The Add22 makes an error of $\epsilon_{Add22}=\epsilon_{103}$\\
Line 7 & We compute $\xred = \dfrac{x-b_i}{1+x.b_i}$. The DIV2 makes $\epsilon_{104}$ (according to Ziv \cite{Ziv91}) error so we have :
\end{tabular}
\bigskip

\begin{equation}
\begin{split}
   \round\big(\xred\big) & = \frac{(x-b_i).(1+\epsilon_{105})}{(1+x.b_i).(1+\epsilon_{105})
          .(1+\epsilon_{105}) )}.(1+\epsilon_{105}) \\
         & =
          \frac{x-b_i}{1+x.b_i}.(1+\epsilon_{105})(1+\epsilon_{105}+\epsilon_{105}+\epsilon_{104})\\
         & = \xred . (1+\epsilon_{101.9})\nonumber
\end{split}
\end{equation}
So: 
\begin{equation}
\epsilon_{\xred} = \epsilon_{102.6} \label{eps:equation}
\end{equation}
\subsubsection{Error about polynomial evaluation}
\begin{lstlisting}[caption={Polynomial Evaluation},firstnumber=1]
      Xred2 = Xredhi*Xredhi;
      
      q = Xred2*(coef_poly[3]+Xred2*
                 (coef_poly[2]+Xred2*
                  (coef_poly[1]+Xred2*
                   coef_poly[0]))) ;

      /* reconstruction : atan(x) = atan(b[i]) + atan(x) */
      double atanlolo = Xredlo+ arctan_table[i][ATAN_BLO].d + Xredhi*q;
      double tmphi2, tmplo2;
      Add12( tmphi2, tmplo2, arctan_table[i][ATAN_BHI].d, Xredhi);
      Add12( atanhi, atanlo, tmphi2, (tmplo2+atanlolo));

\end{lstlisting}
\begin{tabular}{ll}
Line 1 & $\xred \times \xred$ makes an error of $\epsilon_{53}$ so $\xred2 =
      o((\xredhi)^2) = (1+\epsilon_{53}). (\xredhi)^2 = ( 1 + \epsilon_{53}).\big(\round(\xred).(1+\epsilon_{53}+\epsilon_{\xred})\big)^2 = x^2 +  \epsilon_{53} $\\ 
      &the error about x2 is $\epsilon_{51.4}$ \\
Line 3 & Horner approximation with error on x2 :
      Maple compute an error around $\epsilon_{50.7}$ so $\delta_{horner}=\delta_{69.6}$\\ 
Line 5-9 & We need to add $\arctan(b_i)_{hi}+\arctan(b_i)_{lo} +
      (\xredhi+\xredlo).(1+q)$\\
      &
      We compute this in tree steps $test_{lo}= \xredlo+ \arctan(b_i)_{lo}
      + \xredhi.q$.\\
      & then we add $\xredhi+\arctan(b_i))_{hi}$ at the end we add the two results.\\

      &We have two kinds of errors : 
\end{tabular}      
  \begin{itemize} 
    \item Truncation error of  $\xredlo.q$
      but $\xredlo < \xredhi.2^{-53}$ and $q<x^2<2^{-12.6}$ so $\epsilon_{trunc} =
      \epsilon_{65.6}$. 
    \item Rounding error due to the operations : the computation of
      $test_{lo}$ causes 3 errors :\\
       $ \epsilon_{53}.(\xredlo+ \arctan(b_i)_{lo}) < \delta_{100}$\\
       $\epsilon_{53}.\xredhi.q < \delta_{53+6.3+6.3*2} < \delta_{71.9}$\\
       $ \epsilon_{53} . atanlolo < \delta_{65.5}$\\
       $delta_{recons} =  \delta_{71.9}$
      \end{itemize}
      If we add all these errors, we have \\
      $atanlolo_{error} \leq \epsilon_{trunc}.e + \delta_{100} +
      \epsilon_{71.9} + delta_{recons} = \delta_{68.72}$\\

We represent the different values on the next figures :

\begin{center}
 \small
 \setlength{\unitlength}{3ex}
      \framebox{
        \begin{picture}(22,3.5)(-3,-4.15)
         \put(9.5,-0.5){\line(0,-1){4}}  \put(9,-1){$\epsilon$}
  
          \put(4,-2){$\arctan(b_i)_{hi}$} \put(0.05,-2.15){\framebox(7.9,0.7){}}
          \put(12,-2){$\arctan(b_i)_{lo}$}  \put(8.05,-2.15){\framebox(7.9,0.7){}}

          \put(4,-3){$\xredhi$} \put(0.55,-3.15){\framebox(7.9,0.7){}}
          \put(12,-3){$\xredlo$}  \put(8.55,-3.15){\framebox(7.9,0.7){}}

          \put(5,-4){$\xredhi.Q$} \put(2.05,-4.15){\framebox(7.9,0.7){}}
          \put(13,-4){$\xredlo.Q$}  \put(10.05,-4.15){\framebox(7.9,0.7){}}

        \end{picture}
      }
  \end{center}
\begin{center}
 \small
 \setlength{\unitlength}{3ex}
      \framebox{
        \begin{picture}(22,3.5)(-3,-4.15)
         \put(9.5,-0.5){\line(0,-1){4}}  \put(9,-1){$\epsilon$}
  
          \put(4,-2){$\arctan(b_i)_{hi}$} \put(0.05,-2.15){\framebox(7.9,0.7){}}
          \put(12,-2){$\arctan(b_i)_{lo}$}  \put(8.05,-2.15){\framebox(7.9,0.7){}}

          \put(4,-3){$\xredhi$} \put(0.55,-3.15){\framebox(7.9,0.7){}}
          \put(12,-3){$\xredlo$}  \put(8.55,-3.15){\framebox(7.9,0.7){}}

          \put(5,-4){$\xredhi.Q$} \put(2.05,-4.15){\framebox(7.9,0.7){}}

        \end{picture}
      }
  \end{center}
\begin{center}
 \small
 \setlength{\unitlength}{3ex}
      \framebox{
        \begin{picture}(22,2.5)(-3,-4.15)
         \put(9.5,-0.5){\line(0,-1){4}}  \put(9,-1){$\epsilon$}
  
          \put(0.5,-3){$(\arctan(b_i)_{hi} + \xredhi)_{hi}$}
          \put(0.05,-3.15){\framebox(7.9,0.7){}}
          \put(8.5,-3){$(\arctan(b_i)_{hi} + \xredhi)_{lo}$} \put(8.05,-3.15){\framebox(7.9,0.7){}}

          \put(5,-4){$atanlolo$} \put(2.05,-4.15){\framebox(7.9,0.7){}}


        \end{picture}
      }
  \end{center}

\begin{tabular}{ll}
Line 8 & Add12 add no error\\
Line 9 & Add12 add no error
\end{tabular}

\bigskip

\subsubsection {Error due to the polynomial approximation}

The error due to the polynomial approximation is $\delta_{approx} =
\infnorm{ \arctan(x) - x.(1+q)}= \delta_{72,38}$ 

\subsubsection {Final error and rounding constant}

We have to add all error : 
\begin{equation}
\delta_{final} = \delta_{horner} + \delta_{approx} + atanlolo_{error} = \delta_{68}
\end{equation}

So when $i < 10$, the relative error is $\epsilon_{61.7}$ that leads to a
rounding constant of $1.0047$.

And when $i > 10$ the relative error is $\epsilon_{67.8}$ that leads to a
rounding constant of $1.000068$.

\subsubsection{Error when there is no reduction}
\begin{lstlisting}[caption={No reduction},firstnumber=1]

      x2 = x*x;
      q = x2*(coef_poly[3]+x2*
                 (coef_poly[2]+x2*
                  (coef_poly[1]+x2*
                   coef_poly[0]))) ;
      Add12(atanhi,atanlo, x , x*q);

\end{lstlisting}

The code is very simple so there are few errors:

\begin{tabular}{ll}
Line 1 & $\epsilon_{53}$ \\
Line 2 & The Maple procedure to compute Horner approximation gives $\epsilon_{51}$\\
Line 3 & $delta_{no\_reduction} = \epsilon_{105}.x + \epsilon_q.x^3 + 
\epsilon_{x.q}.x^3 + |arctan(x) - x.(1+q)| $
\end{tabular}

When $x>2^{-10}$ the relative error is $\epsilon_{62.9}$. The
constant is $1.0024$. 

When $x<2^{-10}$ the relative error is $\epsilon_{70.4}$. The
constant is $1.0000132$. 

\bigskip

\subsection{Exceptional cases and Rounding}
\subsubsection{Rounding to nearest}
\begin{lstlisting}[caption={Exceptional cases : rounding to nearest},firstnumber=1]

  db_number x_db;
  x_db.d = x;
  unsigned int hx = x_db.i[HI_ENDIAN] & 0x7FFFFFFF; 

  /* Filter cases */
  if ( hx >= 0x43500000)           /* x >= 2^54 */
    {
      if ( (hx & 0x000fffff) == 0x00080000 && x_db.i[LO_ENDIAN] == 0 ) 
        return 0.0/0.0;                /* NaN */
      else
        return HALFPI.d;           /* \arctan(x) = Pi/2 */
    }
  else
    if ( hx < 0x3E400000 )
      {return x;                   /* x<2^-27 then \arctan(x) =~ x */}

\end{lstlisting}
\begin{tabular}{ll}
Lines 3 & Test if x is greatear than $2^{54}$, $\infty$ or $NaN$. \\
Line 5,6 & return $\arctan(NaN) = NaN$\\
Line 8 & \texttt{HALFPI} is the greatest double smaller than
$\frac{\pi}{2}$ in order not to have $\arctan(x) > \dfrac{pi}{2}$ \\
Line 11 & When $x<2^{-27}$ : $x^2 < 2^{-54}$ so $o(\arctan(x)) = x$
\end{tabular}
\\

\textbf{Proof}

 we know that $\arctan(x) = \displaystyle {\sum_{i=0}^{\infty}
\frac{x^{2i+1}}{2i+1}(-1)^i}$.

So:
\begin{equation}
   \begin{split}
       \Big| \frac{\arctan(x)-x}{x}  \Big| & = 
       \Bigg|\frac{ \displaystyle {\sum_{i=0}^{\infty}
       \Big( \frac{x^{2i+1}}{2i+1}(-1)^i} \Big) - x}{x} \Bigg|
       \nonumber\\
       & = \Big|\displaystyle {\sum_{i=1}^{\infty}}
       \frac{x^{2i}}{2i+1}(-1)^i\Big|\nonumber \\ 
       & < \frac{x^2}{3}\nonumber \\
       & < 2^{-54} \nonumber
   \end{split}
\end{equation}

Then : $\arctan(x) \approx x $
\bigskip

\subsubsection{Rounding toward $-\infty$}
\begin{lstlisting}[caption={Exceptional cases : rounding down},firstnumber=1]

  if ( hx >= 0x43500000)           /* x >= 2^54 */
    {
      if (  (hx & 0x000fffff) == 0x00080000 && x_db.i[LO_ENDIAN] == 0)
        return 0.0/0.0;                /* NaN */
      else
        if (x>0)
          return HALFPI.d;
        else
          return -HALFPI_TO_PLUS_INFINITY.d;           /* atan(x) = Pi/2 */
    }
  else
    if ( hx < 0x3E400000 )
      {if (sign>0)
        {if(x==0)
          {x_db.i[HI_ENDIAN]  = 0x80000000;
          x_db.i[LO_ENDIAN] = 0;}
        else
          x_db.l--;
        return x_db.d;
        }
      else
        return x;
      }
  
\end{lstlisting}

The differences with rounding to nearest mode are for $frac{pi}{2}$ for
$x<2^(-27)$.

\begin{lstlisting}[caption={Test for rounding down},firstnumber=1]
  absyh.d=atanhi;
  absyl.d=atanlo;
  
  absyh.l = absyh.l & 0x7fffffffffffffffLL;
  absyl.l = absyl.l & 0x7fffffffffffffffLL;
  u53.l     = (absyh.l & 0x7ff0000000000000LL) +  0x0010000000000000LL;
  u.l   = u53.l - 0x0350000000000000LL;
  
  if(absyl.d > roundcst*u53.d){
    if(atanlo<0.)
      {atanhi -= u.d;}
    return atanhi;
  }
  else {
    return scs_atan_rd(sign*x);
  }
\end{lstlisting}

\subsubsection{Rounding toward $-\infty$}

\begin{lstlisting}[caption={Exceptional cases : rounding up},firstnumber=1]

  if ( hx >= 0x43500000)           /* x >= 2^54 */
    {
      if (   (hx & 0x000fffff) == 0x00080000 && x_db.i[LO_ENDIAN]  == 0)
        return 0.0/0.0;                /* NaN */
      else
        if (x>0)
          return HALFPI.d;
        else
          return -HALFPI_TO_PLUS_INFINITY.d;           /* atan(x) = Pi/2 */
    }
  else
    if ( hx < 0x3E400000 )
      {if (sign<0)
        {x_db.l--;
        return -x_db.d;
        }
      else
        if(x==0)
          return 0;
      return x;
      }

\end{lstlisting}

There are the same differences for rounding down.

\subsubsection{Rounding to zero}

This function is quite simple: it call one of the two function defineded
before.

\begin{lstlisting}[caption={Rounding to zero},firstnumber=1]

extern double atan_rz(double x) {
  if (x>0)
    return atan_rd(x);
  else
    return atan_ru(x);
}
\end{lstlisting}

\subsubsection{Test if rounding is possible}
This test use the theorem \ref{th:roundingRN1}.
The code is the same than in the theorem except that we have 4 rounding
constants : 
\begin{itemize}
\item 1.0047 when $i<10$
\item 1.000068 when $i\geq10$
\item 1.0024 when $x>2^{-10}$
\item 1.0000132 when $x<2^{-10}$
\end{itemize}


\section{Accurate phase}
The accurate phase is the same as the quick phase, except that number are
scs and not double.

The intervals are the same as in quick phase. The only difference is that
$\arctan(b_i)$ as a third double to improve the precision of $\arctan(b_i)$ to
150 bits. Then we use less memory, that is why we can use the same
intervals as in the quick phase.

The polynomial degree is 19 in order to have 136 bits of precision.

\begin{equation} \arctan(x) \approx
x-\frac{1}{3}.x^3+\frac{1}{5}.x^5-\frac{1}{7}.x^7+\frac{1}{9}.x^9-\frac{1}{11}.x^{11}+\frac{1}{13}.x^{13}-\frac{1}{15}.x^{15}+\frac{1}{17}.x^{17}-\frac{1}{19}.x^{19}
\label{eq:arctan_scspoly}
\end{equation}

\section{Analysis of the performance}

\subsection{Speed}
Table \ref{tbl:arctan_abstime} (produced by the \texttt{crlibm\_testperf}
executable) gives absolute timings for a variety of processors. The test
computed 50000 atan in rounding to nearest mode. The second step of
\texttt{crlibm} was taken 1 time on 50000.

\begin{table}[!htb]
\begin{center}
\renewcommand{\arraystretch}{1.2}
\begin{tabular}{|l|r|r|r|}
\hline
\hline

 \multicolumn{4}{|c|}{Pentium 4 Xeon / Linux Debian / gcc 2.95}   \\
 \hline
                         & min time      & max time      & avg time \\ 
 \hline
                         & min time      & max time      & avg time \\ 
 \hline
 \texttt{libm}           & 180          & 344           &        231 \\ 
 \hline
  \texttt{mpfr}          & 417016       & 3956992       &     446362 \\ 
 \hline
  \texttt{libultim}      & 48           & 257616        &        189 \\ 
 \hline
 \texttt{crlibm}         & 36           & 46428         &        381 \\ 
 \hline
 \hline
  \multicolumn{4}{|c|}{PowerPC G4 / MacOS X / gcc2.95}   \\
 \hline
                         & min time      & max time      & avg time \\
 \hline
 \texttt{libm}           & 6            & 11            &          9 \\
 \hline
  \texttt{mpfr}          & 35291        & 303019        &      37022 \\
 \hline
  \texttt{libultim}      & 7            & 13251         &         14 \\
 \hline
 \texttt{crlibm}         & 5            & 1037          &         19 \\
 \hline
 \hline

\end{tabular}
\end{center}
\caption{Absolute timings for the inverse tangent (arbitrary units)
  \label{tbl:arctan_abstime}}
\end{table}

\subsection{Memory requirements}
Table size is
\begin{itemize}
\item for the \quick\ phase,
  $62\times (1+1+2) \times8=1984$ bytes for the 62 $a_i$, $b_i$,
  $\arctan(b_i)$ (hi and lo), plus another $8$ bytes for the rounding
  constant, plus $4\times8$ for the polynomial, $8$ Bytes for
  $\frac{\pi}{2}$ and $64$ for the rounding constants or $2096$ Bytes in
  total.
  
\item for the \accurate\ phase, we just have $10$ SCS constants for the
  polynomial, and 62 other double for $\arctan(b_i)_{lo_{lo}}$.
  If we add all : $10*11*8 + 62*8 = 1376$
\end{itemize}
If we add the fast phase and the acurate one, we have a total of 3472
bytes. 


\section{Conclusion and perspectives}

Our $\arctan$ is reasonably fast.  Libultim is faster but requires ten
times more memory (241 polynomials of degree 7 and 241 of degree 16
that represents more than 40ko !). Instead, our argument reduction
performs a division, which is an expensive operation.

To improve performances we could inline the code of \texttt{atan\_quick},
but we prefer to keep it as it is in order to ease the evolution of the
algorithm.

The main problem of our $\arctan$ is the worst case time which is about 100
times slower than the average time. Thus to improve performances we could
try to use double-extended. A double-extended has a mantissa of 64 bits
which could transform all double-double operation in double-extended
operation. This format number is present in most of the Intel recent
processors.


\newcommand{\middlei}{\mathrm{middle}[i]}


\section{Overview}

The goal is to compute logarithm function so that it provides
correctly rounded result in double precision. The worst-case accuracy
required for this purpose is $118$ bits according to Lef�vre and
Muller \cite{LefMul2004}.

We therefore proceed in two phases \cite{Ziv91}.  The first, quick phase
(program \texttt{log\_fast.c}) is  accurate only to $59-63$
bits.  If this is not enough to decide correct rounding, a second step
accurate to $120$ bits using the SCS library is lauched (program
\texttt{log.c}).


\subsubsection*{Definition interval and exceptional cases}

The natural logarithm is defined over positive floating point numbers.  

\begin{itemize}
\item If $x \le 0$ , then $\log(x)$ should return $NaN$
\item If $x = +\infty$ , then $\log(x)$ should return $+\infty$. 
\end{itemize}

This is true in all rounding modes.

\subsubsection*{Avoiding denormals} 

If $x < 2^{-1022}$ , ie if x is a subnormal number, then we use the equation
$$\log(x) = -52 * \log(2) + \log\left(\frac{x}{2^{-52}}\right)$$ where
$\displaystyle \frac{x}{2^{-52}}$ is now a normalized number.

As $\log(1+\epsilon) \approx \epsilon$ when $\epsilon\rightarrow 0$,
the smallest exponent of a logarithm for a double-precision input
number will be for the input values $\log(1+2^{-52}$ and
$\log(1-2^{-52}$. This ensures that the output will never be a
denormal. This will allow us to ensure that no denormal ever appears
in the computation of the logarithm of a double-precision input
number.




\section{Quick phase}

\subsection{Description of the algorithm}

The algorithm consists of an argument reduction using the well-known
property of the logarithm, and a polynomial evaluation using a degree
12 polynomial.

\subsubsection{Argument reduction and reconsruction}

It is based around the equation 
\begin{equation}
x = 2^{E} * y \label{eq:argred}
\end{equation}
where $E$ is an integer, and $y$ satisfies
\begin{equation}
\frac{11}{16}<\frac{\sqrt{2}}{2} < y < \sqrt{2}<\frac{23}{16} \quad.
\end{equation}

The final reconstruction will then use the equation
 \begin{equation}
\log(x) = E * \log(2) + \log(y) \quad.
\end{equation}

The interval $[\frac{11}{16},\frac{23}{16}]$ being too large for a
polynomial approximation of acceptable degree, it is broken down into
8 intervals given in Table~\ref{table:TablePolysLog1}.  Note that the
first four intervals are of size $2^{-4}$, while the last four are of
size $2^{-3}$.  The value of $i$, the index of the interval $X[i]$ to
which $y$ belongs, will be computed out of a few bits of $y$.

Noting $\middlei$ the middle of the $i$-th interval, the final range
reduction consists in computing $z = y - \middlei$. On each interval, a
polynomial $P[i](z)$ approximates $\log(y)$. In the following $P[i]$
will be noted $P$ when no ambiguity arises.



\begin{table}[htdp]\caption{polynomial precision\label{table:TablePolysLog1}}
\renewcommand{\arraystretch}{1.3}
\begin{center}
\begin{tabular}{|c|c|c|c|c|c|}
\hline
polynomial &   definition interval     &   $\middlei$   & max value of $|z|$ & relative approx. error\\
\hline  
P[0] & $[\frac{11}{16},\frac{12}{16}]$ &  $\frac{23}{32}$ & $2^{-5}$  \\ 
\hline 
P[1] & $[\frac{12}{16},\frac{13}{16}]$ &  $\frac{25}{32}$ & $2^{-5}$  \\
\hline 
P[2] & $[\frac{13}{16},\frac{14}{16}]$ &  $\frac{27}{32}$ & $2^{-5}$  \\ 
\hline 
P[3] & $[\frac{14}{16},\frac{15}{16}]$ &  $\frac{29}{32}$ & $2^{-5}$  \\ 
\hline
P[4] & $[\frac{15}{16},\frac{17}{16}]$ &  $\frac{31}{32}$ & $2^{-4}$  & 61.72 \\ 
\hline 
P[5] & $[\frac{17}{16},\frac{19}{16}]$ &  $\frac{18}{16}$ & $2^{-4}$  \\ 
\hline 
P[6] & $[\frac{19}{16},\frac{21}{16}]$ &  $\frac{20}{16}$ & $2^{-4}$  \\ 
\hline 
P[7] & $[\frac{21}{16},\frac{23}{16}]$ &  $\frac{22}{16}$ & $2^{-4}$  \\ 
\hline
\end{tabular}\end{center}\end{table}

\subsubsection{Polynomial approximation}

On each interval, we have a polynomial $P(z)$ of degree 12 which
approximates $\log(y)$ with an error less than $2^{-60}$.  Each
polynomial has coefficients which are exactly representable as IEEE
doubles, with the two first coefficients being exactly representable
as the sum of two doubles: $c_0 = c_0^{hi} + c_0^{lo}$ and $c1 =
c_1^{hi} + c_1^{lo}$. 


The polynomials are produced by a program in
\texttt{maple/coef\_log.mw}, which directly produces the file
\texttt{log\_fast.h}.


\subsubsection{Reconstruction}

The reconstruction computes: 
$$\log(x) \approx E\times \log(2) + P(z)$$
where $P(z)$ has been
computed by the previous step as the sum of two double-precision
numbers.  The constant $\log(2)$ is also stored as the sum of two
double-precision numbers, and $E$ is a relatively small integer. This
computation uses double-double arithmetic.


\subsubsection{Error analysis}


The polynomials are evaluated thanks to a Horner scheme:

$$P(z) = c_0^{hi}+c_0^{lo} + z .(c_1^{hi} +c_1^{lo} + z .(c_2 + z
  .(c_3 + ...
+ z .(c_{11} + z . (c_{11} + (c_{12} . z))))))))))))
$$
where the two last iterations may use double-double arithmetic if
required by the overall target accuracy, as detailed below.


The reconstruction adds another small error.  As $|E|<1024+52$, we have
$|E|\log(2)<746$, and the maximum absolute error of storing $\log(2)$
as two doubles and multiplying by $E$ is smaller than $2^{-90}$.


The program in \texttt{maple/coef\_log.mw} computes, on each interval,
the maximum approximation error $\deltapprox$ (this is a relative
error), the accumulated rounding error of the Horner scheme
$\deltaround$ (this is an absolute error) and the maximum value of the
polynomial on the interval $\maxp$. This Maple script directly outputs
the values of the rounding constants required by theorems of section
\ref{section:testrounding}.


Error analysis distinguishes four cases, corresponding to four
execution paths in the program.
\begin{enumerate}
\item {If $|E\log(2)| > \mathtt{MIN\_FASTPATH}=128.5$} then
the whole polynomial evaluation may be performed in double-precision,
with a total (approximation and rounding) absolute error smaller than
$2^{-55.7}$. As the absolute value of $P(x)$ in this scheme is always
smaller than $0.38$, this ensures a relative error on $E\log(2)+P(x)$
smaller than $2^{-62.8}$.

\item {If $16.5<|E\log(2)|\le 128.5$} then the last two additions and
the last multiplication of the Horner polynomial evaluation are
performed in double-double-precision. The total relative error in this
case is smaller than $2^{-64.8}$.

\item {If $0<|E\log(2)| < 16.5$}
then we use double-double arithmetic as in the previous case, and the
bound on the relative error is $2^{-60.2}$.

\item {If E=0} then the last two additions and
the last two multiplications of the Horner polynomial evaluation are
performed in double-double-precision. The total relative error in this
case is smaller than $2^{-57.7}$.
\end{enumerate}

The first case represents a trade-off between accuracy (which impacts
the percentage of calls to the \accurate\ phase), and  speed. The last
case is required by the fact that the polynomial corresponding to
input values around $1$ requires a specific rounding error analysis.






\subsection{Details of computer program}

A procedure \texttt{log\_quick} contains the computation shared by the
three functions \texttt{log\_rn}, \texttt{log\_ru} and
\texttt{log\_rd} (the function \texttt{log\_rz} calls either
\texttt{log\_ru} or \texttt{log\_rd}).  This procedures returns an
approximation to the log as two double-precision numbers, and an index
in an array of constants for testing if correct rounding is possible.
This array contains the relative error for directed rounding modes
(see~ref{th:roundingDirected} p.~\pageref{th:roundingDirected}) , and
the rounding constant computed as per Theorem~\ref{th:roundingRN1}
p.~\pageref{th:roundingRN1} for round-to nearest.


\subsubsection{Exceptional cases and argument reduction}

This part  is shown for \texttt{log\_rn}, but it is identical for the three functions.

\newpage
\begin{lstlisting}[caption={Exceptional cases},firstnumber=1]
 double log_rn(double x) { 
   db_number y;
   double res_hi,res_lo,roundcst;
   int E,rndcstindex;

   E=0;
   y.d=x;

   /* Filter cases */
   if (y.i[HI_ENDIAN] < 0x00100000){        /* x < 2^(-1022)    */
     if (((y.i[HI_ENDIAN] & 0x7fffffff)|y.i[LO_ENDIAN])==0){
       return -1.0/0.0;     
     }                                     /* log(+/-0) = -Inf */
     if (y.i[HI_ENDIAN] < 0){ 
       return (x-x)/0;                      /* log(-x) = Nan    */
     }
     /* Subnormal number */
     E = -52;           
     y.d *= two52.d;      /* make x a normal number    */ 
   }
    
   if (y.i[HI_ENDIAN] >= 0x7ff00000){
     return  x+x;                                /* Inf or Nan       */
   }
   
   /* reduce to  y.d such that sqrt(2)/2 < y.d < sqrt(2) */
   E += (y.i[HI_ENDIAN]>>20)-1023;                              /* extract the exponent */
   y.i[HI_ENDIAN] =  (y.i[HI_ENDIAN] & 0x000fffff) | 0x3ff00000;        /* do exponent = 0 */
   if (y.d > SQRT_2){
     y.d *= 0.5;
     E++;
   }

   /* Call the actual computation */
   log_quick(&res_hi, &res_lo, &rndcstindex, &y, E);
\end{lstlisting}


\begin{tabular}{ll}
Lines  6,7 &  Initialize E and y\\
Line 10 & Test if x is null, negative or a subnormal number.\\
Line 11,12 & Test if x is $\pm 0$ and return  $-\infty$ \\
Line 14,15 & If x is negative,  return NaN and raise an exception.\\
Line 18,19 & else $x$ is subnormal, then compute:\\
& $log(x) = -52\times log(2) + log(\frac{x}{2^{-52}})$ \\
& (this computation is exact) \\ 
Line 22,23 & If $x$ is $\infty$ or NaN, return $\infty$ or NaN.\\
Line 27 & E contains x 's exponent. \\
Line 28 & y.d is reduced to $\frac{x}{2^E}$. Correct because $y$ was a normal number.\\
Line 29-31 & Now, we have: $ 1 \leq y.d < 2$ and we want $ \frac{1}{\sqrt2} \leq y.d < \sqrt2$.\\
& So, if it's not the case, we update E and y.\\
\end{tabular}


\newpage
\begin{lstlisting}[caption={Procedure \texttt{log\_quick}},firstnumber=1]
static void log_quick(double *pres_hi, double *pres_lo, int* prndcstindex, db_number * py, int E) {
   double ln2_times_E_HI, ln2_times_E_LO, res_hi, res_lo;
   double z, res, P_hi, P_lo;
   int k, i;
   
    /* find the interval including y.d */
    i = ((((*py).i[HI_ENDIAN] & 0x001F0000)>>16)-6) ;
    if (i < 10)
      i = i>>1;
    else
      i = ((i-1)>>1);
    
    z = (*py).d - (middle[i]).d;  /* (exact thanks to Sterbenz Lemma) */
    

    /* Compute ln2_times_E = E*log(2)   in double-double */
    Mul22(&ln2_times_E_HI, &ln2_times_E_LO, ln2hi.d, ln2lo.d, (double)E, 0.);


    /* Now begin the polynomial evaluation of log(1 + z)      */

    res = (Poly_h[i][DEGREE]).d;

    for(k=DEGREE-1; k>1; k--){
      res *= z;
      res += (Poly_h[i][k]).d;
    }

    if((ln2_times_E_HI*ln2_times_E_HI < MIN_FASTPATH*MIN_FASTPATH)) {
      /* Slow path */
      if(E==0) {
        *prndcstindex = 0 ;
        /* In this case we start with a double-double multiplication to get enough relative accuracy */ 
        Mul12(&P_hi, &P_lo, res, z); 
        Add22(&res_hi, &res_lo, (Poly_h[i][1]).d,  (Poly_l[i][1]).d, P_hi, P_lo);
        Mul22(&P_hi, &P_lo, res_hi, res_lo, z, 0.); 
        Add22(pres_hi, pres_lo, (Poly_h[i][0]).d, (Poly_l[i][0]).d, P_hi, P_lo);
      } 
      else
        {
          if((ln2_times_E_HI*ln2_times_E_HI > 16.5*16.5))
            *prndcstindex = 2; 
          else 
            *prndcstindex =1;
          P_hi=res*z;  P_lo=0.; 
          Add22(&res_hi, &res_lo, (Poly_h[i][1]).d,  (Poly_l[i][1]).d, P_hi, P_lo);
          Mul22(&P_hi, &P_lo, res_hi, res_lo, z, 0.); 
          Add22(&res_hi, &res_lo, (Poly_h[i][0]).d, (Poly_l[i][0]).d, P_hi, P_lo);
      
        /* Add E*log(2)  */
          Add22(pres_hi, pres_lo, ln2_times_E_HI, ln2_times_E_LO, res_hi, res_lo);
        }
    }
    else { /* Fast path */
      
      *prndcstindex = 3 ;
      res =   z*((Poly_h[i][1]).d + z*res);
      Add22(&res_hi, &res_lo, (Poly_h[i][0]).d , (Poly_l[i][0]).d, res, 0.);

        /* Add E*log(2)  */
      Add22(pres_hi, pres_lo, ln2_times_E_HI, ln2_times_E_LO, res_hi, res_lo);
    }
}
\end{lstlisting}

\begin{tabular}{ll}
Line 7  & To find the interval $X_i$ containing $y$, we  look upon\\ 
        &   the last bit of the exponent and the 4 first bits of the mantissa.\\
Lines 8-11  & Reduction over $i$ in order to have an index $i$ between 0 and 7\\
        & corresponding to the 8 intervals of Table~\ref{table:TablePolysLog1}.\\
Line 13 & Let us prove that z.d is computed exactly, without rounding error: \\
& \vspace{1ex}To use Sterbenz Lemma, we need to prove that  ${\middlei}/{2} <  y.d < \middlei \times 2$.\\ 
& \vspace{1ex}For every i in $[0;7]$, we have $\middlei-\frac{1}{16} \leq y.d \leq \middlei+\frac{1}{16}$.\\
& \vspace{1ex}Table~\ref{table:TablePolysLog1} gives: $\frac{25}{32} \leq {\middlei} \leq \frac{22}{16}$\\ 
& \vspace{1ex}Therefore ${\middlei}/{2} \leq \frac{22}{32} < \frac{23}{32} \leq \middlei - \frac{1}{16} \leq y.d$\\
& and $y.d \leq \middlei + \frac{1}{16} \leq \frac{23}{16} < \frac{25}{16} \leq \middlei\times 2$\\
\end{tabular}



\subsection{Rounding}




\section{Accurate phase}


The function called is \texttt{scs\_log\_rn} for a result rounded to
nearest, \texttt{scs\_log\_rd} for rounding down, or
\texttt{scs\_log\_ru} for rounding up.



\subsection{Argument reduction}

Argument reduction is the same as in the first step: 
$x = 2^E \times y$,
with $y = 1 + f$, $f \leq 2^{-1}$
and  $\frac{1}{\sqrt(2)} \leq y < sqrt(2)$

Therefore the \texttt{scs\_log\_*} functions take as arguments $y$ and $E$, computed
in the first step (similarly, exceptional cases are not considered
again).

As in the first step we will compute the log using $$\log(x) = E \times log(2) + log(1+f)\quad .$$

Now we define $w_i = 1 + i\times2^{-4}$, for $i = -6 ... 6$, and we select the $w_i$ closest to $1+f$, in order to have: 

$$log(1+f) = log(w_i) + log(1+\frac{1+f-w_i}{w_i})$$

where $r=\frac{1+f-w_i}{w_i} \leq 2^{-5}$.


\begin{lstlisting}[caption={Argument reduction},firstnumber=1]
 /* to normalize y.d and round to nearest      */
  /* + (1-trunc(sqrt(2.)/2 * 2^(4))*2^(-4) )+2.^(-(4+1))*/ 
  z.d = y.d + norm_number.d; 
  i = (z.i[HI_ENDIAN] & 0x000fffff);
  i = i >> 16; /* 0<= i <=11 */
  

  wi.d = (11+i)*(double)0.6250e-1;

  /* (1+f-w_i) */
  y.d -= wi.d; 
  
  /* Table reduction */
  ti     = table_ti_ptr[i]; 
  inv_wi = table_inv_wi_ptr[i];
   
  /* R = (1+f-w_i)/w_i */
  scs_set_d(R, y.d);
  scs_mul(R, R, inv_wi);

\end{lstlisting}

\begin{tabular}{ll}
Lines 3-5 & We compute the "i" corresponding to z, through the four first bits of z's mantissa.\\
Line 7 & $w_i$ is computed.\\
Lines 14-15 & We get tabulated values: $t_i = log(w_i)$ and $\frac{1}{w_i}$\\
Lines 18-19 & We compute the polynomial argument: $R = \frac{1+f-w_i}{w_i}$.\\ 
\end{tabular}



\subsection{Polynomial approximation}
$log(1+\frac{1+f-w_i}{w_i})$ is approximated by  a polynomial $Q(\frac{1+f-w_i}{w_i})$ with an error less than $2^{-130}$.

$log(w_i)$ and $log(2)$ are tabulated. The polynomials are given in appendix


The computation is similar as the first step computation, except that everything is computed in SCS format, it means that every computation is realized with 211 bits precision.
so it is clearly enough to reach a result with 130 bits precision.\\



\subsection{Reconstruction}
So, at the end, we compute:\\
\begin{equation}result = E\times log(2) + log(w_i) + Q(\frac{1+f-w_i}{w_i})\end{equation}


\subsection{Rounding mode}
The procedures $scs\_get\_d$, $scs\_get\_d\_pinf$ and $scs\_get\_d\_minf$ assure the conversion from type "scs" to type "IEEE" according to the selected round mode.\\
You are encouraged for further details to have a look at the research report RR 2003-37.pdf at www.ens-lyon.fr/LIP/Pub/rr2003.html




%%%%%%%%%%%%%%%%%%%%%%%%%%%%%%%%%%%%%%%%%%%%%%%%%%%%%%%%%%%%%
\section{Analysis of the logarithm performance}
\label{section:log_results}

Table \ref{tbl:log_abstime} (produced by the \texttt{crlibm\_testperf}
executable) gives absolute timings for a variety of processors and
operating systems. Contributions to this table for new
processors/OS/compiler combinations are welcome.

\begin{table}[!htb]
\begin{center}
\renewcommand{\arraystretch}{1.2}
\begin{tabular}{|l|r|r|r||r|}
\hline\hline
 \multicolumn{4}{|c|}{Pentium 4 Xeon / Debian sarge / gcc 3.3}   \\ 
 \hline
                         & min time      & max time      & avg time \\ 
 \hline
 \texttt{libm}           & 724          & 7088          &        732 \\ 
 \hline
  \texttt{mpfr}          & 568          & 290924        &      83827 \\ 
 \hline
  \texttt{libultim}      & 784          & 485320        &       1062 \\ 
 \hline\hline
 \texttt{crlibm}         & 968          & 51392         &       1147 \\ 
 \hline
 \hline
\end{tabular}
\end{center}
\caption{Absolute timings for the logarithm
  \label{tbl:log_abstime}}
\end{table}


In average, the second step is taken in 0.23\% of the calls, which
seems a rather good balance considering the respective costs of the
first and second steps (seen in the table as the min and max times,
respectively).

Table size is $8\times 15\times8=960$ bytes for the eight polynomials,
plus another $64$ bytes for the rounding constants, or a total of
exactly 1kB.


%%%%%%%%%%%%%%%%%%%%%%%%%%%%%%%%%%%%%%%%%%%%%%%%%%%%%%%%%%%%%
\section{Conclusion and perspectives}


In the log we have a fairly good balance between both evaluation
phases. The IBM library has specific code for the cases when $x$ is
close to $1$. By following this approach we could hope for a further
50\% performance improvement in average, at the cost of a more
complicated proof.






%% \section{Second step polynomial coefficients.}

%% \begin{table}
%% \caption{Polynomial P 0} 
%% \begin{tabular}{|c|c|c|c|}
%% \hline &&& \\
%% coeff n� & exponent & mantissa & binary number \\ 
%% &&&\\ \hline &&& \\ 
%% 0 HI &$ -2 $ & $\frac{-743638168966267}{562949953421312}$ & $ -.10101001000101010111000000111001110001010001111011000e-1 $ \\ 
%% &&&\\ \hline &&& \\ 
%% 0 LO &$ -57 $ & $\frac{7027957893218633}{4503599627370496}$ & $ .11000111101111110001111110101101110001100000101001000e-56 $ \\ 
%% &&&\\ \hline &&& \\ 
%% 1 HI &$ 0 $ & $\frac{1566469435607129}{1125899906842624}$ & $ 1.0110010000101100100001011001000010110010000101100100 $ \\ 
%% &&&\\ \hline &&& \\ 
%% 1 LO &$ -55 $ & $\frac{3075156854129589}{2251799813685248}$ & $ .10101110110011010110101111111110101101000011101101001e-54 $ \\ 
%% &&&\\ \hline &&& \\ 
%% 2 &$ -1 $ & $\frac{-8717742945987501}{4503599627370496}$ & $ -.11110111110001011110110110011100010011110101110101101 $ \\ 
%% &&&\\ \hline &&& \\ 
%% 3 &$ -1 $ & $\frac{8086022442655257}{4503599627370496}$ & $ .11100101110100011000111001111010101110001111000011001 $ \\ 
%% &&&\\ \hline &&& \\ 
%% 4 &$ -1 $ & $\frac{-1054698579476787}{562949953421312}$ & $ -.11101111110011111000100110001011001100000100110011000 $ \\ 
%% &&&\\ \hline &&& \\ 
%% 5 &$ 0 $ & $\frac{4695701500917009}{4503599627370496}$ & $ 1.0000101011101011011100110011111011101111010100010001 $ \\ 
%% &&&\\ \hline &&& \\ 
%% 6 &$ 0 $ & $\frac{-5444291595459813}{4503599627370496}$ & $ -1.0011010101111000110111101010100110011101110011100101 $ \\ 
%% &&&\\ \hline &&& \\ 
%% 7 &$ 0 $ & $\frac{3246286994737197}{2251799813685248}$ & $ 1.0111000100001111011000101110010100100001000001011010 $ \\ 
%% &&&\\ \hline &&& \\ 
%% 8 &$ 0 $ & $\frac{-1976000318188175}{1125899906842624}$ & $ -1.1100000101001010010110110100001010101010101000111100 $ \\ 
%% &&&\\ \hline &&& \\ 
%% 9 &$ 1 $ & $\frac{4885240473419053}{4503599627370496}$ & $ 10.001010110110001100110101111000011110111110100101101 $ \\ 
%% &&&\\ \hline &&& \\ 
%% 10 &$ 1 $ & $\frac{-382458832789537}{281474976710656}$ & $ -10.101101111011000001000101101111010100000001000010000 $ \\ 
%% &&&\\ \hline &&& \\ 
%% 11 &$ 2 $ & $\frac{299676142110095}{281474976710656}$ & $ 100.01000010001101110010011111111110000101100011110000 $ \\ 
%% &&&\\ \hline &&& \\ 
%% 12 &$ 2 $ & $\frac{-5063189123827543}{4503599627370496}$ & $ -100.01111111001111000110010101001010000000011101010111 $ \\ 
%% &&&\\ \hline &&& \\ 
%% 13 &$ 7 $ & $\frac{-8903658134248167}{4503599627370496}$ & $ -11111101.000011101010001110100010010110001101011100111 $ \\ 
%% &&&\\ \hline
%% \end{tabular}
%% \end{table}

%% \begin{table}
%% \caption{Polynomial P 1} 
%% \begin{tabular}{|c|c|c|c|}
%% \hline &&&\\
%% coeff n�& exponent & mantissa & binary number \\ 
%% &&&\\ \hline &&& \\ 
%% 0 HI &$ -3 $ & $\frac{-8894071639880569}{4503599627370496}$ & $ -.11111100110010001110001101100101100111011001101111000e-2 $ \\ 
%% &&&\\ \hline &&& \\ 
%% 0 LO &$ -57 $ & $\frac{-552389204741253}{281474976710656}$ & $ -.11111011001100101001000100101001110001100100001001111e-56 $ \\ 
%% &&&\\ \hline &&& \\ 
%% 1 HI &$ 0 $ & $\frac{5764607523034235}{4503599627370496}$ & $ 1.0100011110101110000101000111101011100001010001111011 $ \\ 
%% &&&\\ \hline &&& \\ 
%% 1 LO &$ -56 $ & $\frac{-4361576903664429}{2251799813685248}$ & $ -.11110111111011010100110000100000000011100011001011001e-55 $ \\ 
%% &&&\\ \hline &&& \\ 
%% 2 &$ -1 $ & $\frac{-7378697629483821}{4503599627370496}$ & $ -.11010001101101110001011101011000111000100001100101101 $ \\ 
%% &&&\\ \hline &&& \\ 
%% 3 &$ -1 $ & $\frac{3148244321913127}{2251799813685248}$ & $ .10110010111101001111110000000111100101001001001001110 $ \\ 
%% &&&\\ \hline &&& \\ 
%% 4 &$ -1 $ & $\frac{-6044629098073239}{4503599627370496}$ & $ -.10101011110011000111011100010001100001000110010010111 $ \\ 
%% &&&\\ \hline &&& \\ 
%% 5 &$ -1 $ & $\frac{1547425048982465}{1125899906842624}$ & $ .10101111111010111111111100001011100011100111100000100 $ \\ 
%% &&&\\ \hline &&& \\ 
%% 6 &$ -1 $ & $\frac{-3301173437887807}{2251799813685248}$ & $ -.10111011101001100110010101100001101101110101001111110 $ \\ 
%% &&&\\ \hline &&& \\ 
%% 7 &$ -1 $ & $\frac{7243719460619263}{4503599627370496}$ & $ .11001101111000001111111011101010011011000001111111111 $ \\ 
%% &&&\\ \hline &&& \\ 
%% 8 &$ -1 $ & $\frac{-4056482338782501}{2251799813685248}$ & $ -.11100110100101011001011001001101011100111101001001010 $ \\ 
%% &&&\\ \hline &&& \\ 
%% 9 &$ 0 $ & $\frac{144183934662417}{140737488355328}$ & $ 1.0000011001000100111000000110101001101110001000100000 $ \\ 
%% &&&\\ \hline &&& \\ 
%% 10 &$ 0 $ & $\frac{-664564121389669}{562949953421312}$ & $ -1.0010111000110101011100100011000101100101001100101000 $ \\ 
%% &&&\\ \hline &&& \\ 
%% 11 &$ 0 $ & $\frac{7406777826722191}{4503599627370496}$ & $ 1.1010010100000110110011011001010010101100000110001111 $ \\ 
%% &&&\\ \hline &&& \\ 
%% 12 &$ 0 $ & $\frac{-1853617081185135}{1125899906842624}$ & $ -1.1010010101110110101101101110101100011100110110111100 $ \\ 
%% &&&\\ \hline &&& \\ 
%% 13 &$ 6 $ & $\frac{-1467725186095279}{1125899906842624}$ & $ -1010011.0110111000110110011001011000010110001010111100 $ \\ 
%% &&&\\ \hline &&& \\ 
%% \end{tabular}
%% \end{table}

%% \begin{table}
%% \caption{Polynomial P 2} 
%% \begin{tabular}{|c|c|c|c|}
%% \hline &&& \\
%% coeff n�& exponent & mantissa & binary number \\ 
%% &&&\\ \hline &&& \\ 
%% 0 HI &$ -3 $ & $\frac{-3060628955209433}{2251799813685248}$ & $ -.10101101111110100000001101011010101000011110110110001e-2 $ \\ 
%% &&&\\ \hline &&& \\ 
%% 0 LO &$ -61 $ & $\frac{2527594736042079}{2251799813685248}$ & $ .10001111101011010101100000100110000011000000010111110e-60 $ \\ 
%% &&&\\ \hline &&& \\ 
%% 1 HI &$ 0 $ & $\frac{667199944795629}{562949953421312}$ & $ 1.0010111101101000010010111101101000010010111101101000 $ \\ 
%% &&&\\ \hline &&& \\ 
%% 1 LO &$ -54 $ & $\frac{1332688516023975}{1125899906842624}$ & $ .10010111100000100101011100110011000000001101010011100e-53 $ \\ 
%% &&&\\ \hline &&& \\ 
%% 2 &$ -1 $ & $\frac{-6326043921025223}{4503599627370496}$ & $ -.10110011110011000000011100000101111110000100011000111 $ \\ 
%% &&&\\ \hline &&& \\ 
%% 3 &$ -1 $ & $\frac{2499177845343309}{2251799813685248}$ & $ .10001110000011111101001011111011001111000100010011010 $ \\ 
%% &&&\\ \hline &&& \\ 
%% 4 &$ -2 $ & $\frac{-8885965672331789}{4503599627370496}$ & $ -.11111100100011011110100011011011000101011101000001100e-1 $ \\ 
%% &&&\\ \hline &&& \\ 
%% 5 &$ -2 $ & $\frac{8425211896372323}{4503599627370496}$ & $ .11101111011101010111110000010001111011001110001100011e-1 $ \\ 
%% &&&\\ \hline &&& \\ 
%% 6 &$ -2 $ & $\frac{-8321196934766279}{4503599627370496}$ & $ -.11101100100000001010110100011011001110001111011000111e-1 $ \\ 
%% &&&\\ \hline &&& \\ 
%% 7 &$ -2 $ & $\frac{8453280679506371}{4503599627370496}$ & $ .11110000010000011011011001000000101101011000111000011e-1 $ \\ 
%% &&&\\ \hline &&& \\ 
%% 8 &$ -2 $ & $\frac{-4383182203135729}{2251799813685248}$ & $ -.11111001001001111011001000100011010000100110111100010e-1 $ \\ 
%% &&&\\ \hline &&& \\ 
%% 9 &$ -1 $ & $\frac{4616603552660769}{4503599627370496}$ & $ .10000011001101100011011000110110100110011110100100001 $ \\ 
%% &&&\\ \hline &&& \\ 
%% 10 &$ -1 $ & $\frac{-4925251927700411}{4503599627370496}$ & $ -.10001011111110111110110010011100101000111001110111011 $ \\ 
%% &&&\\ \hline &&& \\ 
%% 11 &$ -1 $ & $\frac{6183144258300187}{4503599627370496}$ & $ .10101111101111000100101110101000111000110100100011011 $ \\ 
%% &&&\\ \hline &&& \\ 
%% 12 &$ -1 $ & $\frac{-366830079310573}{281474976710656}$ & $ -.10100110110100001010010001001000111111010111011010000 $ \\ 
%% &&&\\ \hline &&& \\ 
%% 13 &$ 4 $ & $\frac{-2107375484601665}{1125899906842624}$ & $ -11101.111100101001011001011100101011000001010100000100 $ \\ 
%% &&&\\ \hline 
%% \end{tabular}
%% \end{table}


%% \begin{table}
%% \caption{Polynomial P 3} 
%% \begin{tabular}{|c|c|c|c|}
%% \hline &&& \\
%% coeff n�& exponent & mantissa & binary number \\ 
%% &&&\\ \hline &&& \\ 
%% 0 HI &$ -4 $ & $\frac{-7093354803841417}{4503599627370496}$ & $ -.11001001100110101111001011101010110010100100110001000e-3 $ \\ 
%% &&&\\ \hline &&& \\ 
%% 0 LO &$ -58 $ & $\frac{180242050465785}{140737488355328}$ & $ .10100011111011011101111101100100001111111111100011111e-57 $ \\ 
%% &&&\\ \hline &&& \\ 
%% 1 HI &$ 0 $ & $\frac{2484744621997515}{2251799813685248}$ & $ 1.0001101001111011100101100001000110100111101110010110 $ \\ 
%% &&&\\ \hline &&& \\ 
%% 1 LO &$ -56 $ & $\frac{4960769888924131}{4503599627370496}$ & $ .10001100111111100101100111111000010101011000111100011e-55 $ \\ 
%% &&&\\ \hline &&& \\ 
%% 2 &$ -1 $ & $\frac{-5483574338201415}{4503599627370496}$ & $ -.10011011110110100100000100100100001110010011101000111 $ \\ 
%% &&&\\ \hline &&& \\ 
%% 3 &$ -2 $ & $\frac{1008473441508307}{562949953421312}$ & $ .11100101010011001110000101001111110001101111010011000e-1 $ \\ 
%% &&&\\ \hline &&& \\ 
%% 4 &$ -2 $ & $\frac{-3338394840820967}{2251799813685248}$ & $ -.10111101110001000000100111101001101101101100111001110e-1 $ \\ 
%% &&&\\ \hline &&& \\ 
%% 5 &$ -2 $ & $\frac{5893993650002925}{4503599627370496}$ & $ .10100111100001000111001010101110100010001101111101100e-1 $ \\ 
%% &&&\\ \hline &&& \\ 
%% 6 &$ -2 $ & $\frac{-2709882287004295}{2251799813685248}$ & $ -.10011010000010011111101000110001100110011010100001110e-1 $ \\ 
%% &&&\\ \hline &&& \\ 
%% 7 &$ -2 $ & $\frac{5126082608473109}{4503599627370496}$ & $ .10010001101100010010100011100111101111010010000010100e-1 $ \\ 
%% &&&\\ \hline &&& \\ 
%% 8 &$ -2 $ & $\frac{-2474366161612937}{2251799813685248}$ & $ -.10001100101001101100010001000001101101110000100010010e-1 $ \\ 
%% &&&\\ \hline &&& \\ 
%% 9 &$ -2 $ & $\frac{606794337740459}{562949953421312}$ & $ .10001001111110000001010000111100111110011010101011000e-1 $ \\ 
%% &&&\\ \hline &&& \\ 
%% 10 &$ -2 $ & $\frac{-2678423821842551}{2251799813685248}$ & $ -.10011000010000000011001001000111010101111000011101110e-1 $ \\ 
%% &&&\\ \hline &&& \\ 
%% 11 &$ -2 $ & $\frac{4901274782692199}{4503599627370496}$ & $ .10001011010011010111011110110010111100111001101100111e-1 $ \\ 
%% &&&\\ \hline &&& \\ 
%% 12 &$ 3 $ & $\frac{5562257110400349}{4503599627370496}$ & $ 1001.1110000101101011111100111001101101111110101011101 $ \\ 
%% &&&\\ \hline &&& \\ 
%% 13 & $ 0 $ & $ 0 $ & $ 0 $ \\ 
%% &&&\\ \hline 
%% \end{tabular}
%% \end{table}



%% \begin{table}
%% \caption{Polynomial P 4} 
%% \begin{tabular}{|c|c|c|c|}
%% \hline &&& \\
%% coeff n�& exponent & mantissa & binary number \\ 
%% &&&\\ \hline &&& \\ 
%% 0 HI & $ 0 $ & $ 0 $ & $ 0 $ \\ 
%% &&&\\ \hline &&& \\ 
%% 0 LO & $ 0 $ & $ 0 $ & $ 0 $ \\ 
%% &&&\\ \hline &&& \\ 
%% 1 HI & $ 0 $ & $ 1 $ & $ 1. $ \\ 
%% &&&\\ \hline &&& \\ 
%% 1 LO &$ -73 $ & $\frac{4866424671436317}{4503599627370496}$ & $ .10001010010011111110011001001110111011111111000011100e-72 $ \\ 
%% &&&\\ \hline &&& \\ 
%% 2 & $ -1 $ & $ -1 $ & $ -.10000000000000000000000000000000000000000000000000000 $ \\ 
%% &&&\\ \hline &&& \\ 
%% 3 &$ -2 $ & $\frac{6004799503160661}{4503599627370496}$ & $ .10101010101010101010101010101010101010101010101010100e-1 $ \\ 
%% &&&\\ \hline &&& \\ 
%% 4 &$ -2 $ & $\frac{-4503599627370601}{4503599627370496}$ & $ -.10000000000000000000000000000000000000000000001101000e-1 $ \\ 
%% &&&\\ \hline &&& \\ 
%% 5 &$ -3 $ & $\frac{7205759403793217}{4503599627370496}$ & $ .11001100110011001100110011001100110011001101101000000e-2 $ \\ 
%% &&&\\ \hline &&& \\ 
%% 6 &$ -3 $ & $\frac{-1501199875682485}{1125899906842624}$ & $ -.10101010101010101010101010101010011101100001011010100e-2 $ \\ 
%% &&&\\ \hline &&& \\ 
%% 7 &$ -3 $ & $\frac{5146971002059239}{4503599627370496}$ & $ .10010010010010010010010010010001111110011100111100111e-2 $ \\ 
%% &&&\\ \hline &&& \\ 
%% 8 &$ -3 $ & $\frac{-1125900001289929}{1125899906842624}$ & $ -.10000000000000000000000010110100001001001101100100100e-2 $ \\ 
%% &&&\\ \hline &&& \\ 
%% 9 &$ -4 $ & $\frac{8006400287554951}{4503599627370496}$ & $ .11100011100011100011101010101000100011011110110000110e-3 $ \\ 
%% &&&\\ \hline &&& \\ 
%% 10 &$ -4 $ & $\frac{-3602718697199167}{2251799813685248}$ & $ -.11001100110010100111010100000010110000010110001111110e-3 $ \\ 
%% &&&\\ \hline &&& \\ 
%% 11 &$ -4 $ & $\frac{818791763735267}{562949953421312}$ & $ .10111010001010111111010000001110001111001011100011000e-3 $ \\ 
%% &&&\\ \hline &&& \\ 
%% 12 &$ -4 $ & $\frac{-6070595428018661}{4503599627370496}$ & $ -.10101100100010010110010100100000011111110100111100100e-3 $ \\ 
%% &&&\\ \hline &&& \\ 
%% 13 &$ -4 $ & $\frac{2804528659819663}{2251799813685248}$ & $ .10011111011010110100001101101000011111101100100011110e-3 $ \\ 
%% &&&\\ \hline
%% \end{tabular}
%% \end{table}


%% \begin{table}
%% \caption{Polynomial P 5} 
%% \begin{tabular}{|c|c|c|c|}
%% \hline &&& \\
%% coeff n�& exponent & mantissa & binary number \\ 
%% &&&\\ \hline &&& \\ 
%% 0 HI &$ -4 $ & $\frac{4243581083941235}{2251799813685248}$ & $ .11110001001110000011101101110001010101111001011100110e-3 $ \\ 
%% &&&\\ \hline &&& \\ 
%% 0 LO &$ -60 $ & $\frac{-194419322244045}{140737488355328}$ & $ -.10110000110100101100011010100011110010111100110100000e-59 $ \\ 
%% &&&\\ \hline &&& \\ 
%% 1 HI &$ -1 $ & $\frac{2001599834386887}{1125899906842624}$ & $ .11100011100011100011100011100011100011100011100011100 $ \\ 
%% &&&\\ \hline &&& \\ 
%% 1 LO &$ -55 $ & $\frac{6780843324223619}{4503599627370496}$ & $ .11000000101110010010000011101100111100011000010000011e-54 $ \\ 
%% &&&\\ \hline &&& \\ 
%% 2 &$ -2 $ & $\frac{-222399981598543}{140737488355328}$ & $ -.11001010010001011000011111100110101101110100111100000e-1 $ \\ 
%% &&&\\ \hline &&& \\ 
%% 3 &$ -3 $ & $\frac{4217362614017791}{2251799813685248}$ & $ .11101111101110101011010000000111111101011110111111110e-2 $ \\ 
%% &&&\\ \hline &&&\\ 
%% 4 &$ -3 $ & $\frac{-2811575076012255}{2251799813685248}$ & $ -.10011111110100011100110101011010101001000000110111110e-2 $ \\ 
%% &&&\\ \hline &&&\\ 
%% 5 &$ -4 $ & $\frac{3998684548554481}{2251799813685248}$ & $ .11100011010011001000010011000001010111011110111100001e-3 $ \\ 
%% &&&\\ \hline &&&\\ 
%% 6 &$ -4 $ & $\frac{-2961988555123393}{2251799813685248}$ & $ -.10101000010111101001101100111010100000101010110000001e-3 $ \\ 
%% &&&\\ \hline &&&\\ 
%% 7 &$ -4 $ & $\frac{4513513385732155}{4503599627370496}$ & $ .10000000010010000010000111010000010111010100000111011e-3 $ \\ 
%% &&&\\ \hline &&&\\ 
%% 8 &$ -5 $ & $\frac{-7021014623365341}{4503599627370496}$ & $ -.11000111100011001001101011010100101101010100011011101e-4 $ \\ 
%% &&&\\ \hline &&&\\ 
%% 9 &$ -5 $ & $\frac{5541480771625565}{4503599627370496}$ & $ .10011101011111111001010001000111011000010001001011101e-4 $ \\ 
%% &&&\\ \hline &&&\\ 
%% 10 &$ -6 $ & $\frac{-138651326119215}{70368744177664}$ & $ -.11111100001101001000111000010100111110100101111000000e-5 $ \\ 
%% &&&\\ \hline &&&\\ 
%% 11 &$ -5 $ & $\frac{2405838712720693}{2251799813685248}$ & $ .10001000110000011000111110011101101111001101001101001e-4 $ \\ 
%% &&&\\ \hline &&&\\ 
%% 12 &$ -6 $ & $\frac{-3033201869659221}{2251799813685248}$ & $ -.10101100011010101110010101111111100100011100010101010e-5 $ \\ 
%% &&&\\ \hline &&&\\ 
%% 13 &$ -1 $ & $\frac{-5882258564419063}{4503599627370496}$ & $ -.10100111001011110001000001100001101011000110111110111 $ \\ 
%% &&&\\ \hline 
%% \end{tabular}
%% \end{table}


%% \begin{table}
%% \caption{Polynomial P 6} 
%% \begin{tabular}{|c|c|c|c|}
%% \hline &&& \\
%% coeff n�& exponent & mantissa & binary number \\ 
%% &&&\\ \hline &&& \\ 
%% 0 HI &$ -3 $ & $\frac{4019796858195217}{2251799813685248}$ & $ .11100100011111111011111000111100110101001101000100001e-2 $ \\ 
%% &&&\\ \hline &&& \\ 
%% 0 LO &$ -57 $ & $\frac{-2950359927797621}{2251799813685248}$ & $ -.10100111101101010110001100110111010010110111011101001e-56 $ \\ 
%% &&&\\ \hline &&& \\ 
%% 1 HI &$ -1 $ & $\frac{3602879701896397}{2251799813685248}$ & $ .11001100110011001100110011001100110011001100110011010 $ \\ 
%% &&&\\ \hline &&& \\ 
%% 1 LO &$ -55 $ & $\frac{-3752959905019767}{2251799813685248}$ & $ -.11010101010101001100000100011111111010110111011101110e-54 $ \\ 
%% &&&\\ \hline &&& \\ 
%% 2 &$ -2 $ & $\frac{-5764607523034235}{4503599627370496}$ & $ -.10100011110101110000101000111101011100001010001111011e-1 $ \\ 
%% &&&\\ \hline &&& \\ 
%% 3 &$ -3 $ & $\frac{6148914691236995}{4503599627370496}$ & $ .10101110110000110011111000011111011001110010010000011e-2 $ \\ 
%% &&&\\ \hline &&& \\ 
%% 4 &$ -4 $ & $\frac{-7378697629484743}{4503599627370496}$ & $ -.11010001101101110001011101011000111000100011011000110e-3 $ \\ 
%% &&&\\ \hline &&& \\ 
%% 5 &$ -4 $ & $\frac{4722366480912847}{4503599627370496}$ & $ .10000110001101111011110100000100110000001000111001111e-3 $ \\ 
%% &&&\\ \hline &&& \\ 
%% 6 &$ -5 $ & $\frac{-393530540111015}{281474976710656}$ & $ -.10110010111101001111110000000110100110100101001110000e-4 $ \\ 
%% &&&\\ \hline &&& \\ 
%% 7 &$ -6 $ & $\frac{4317595647888997}{2251799813685248}$ & $ .11110101011011010100100100101110000001001110011001010e-5 $ \\ 
%% &&&\\ \hline &&& \\ 
%% 8 &$ -6 $ & $\frac{-6044631180159527}{4503599627370496}$ & $ -.10101011110011000111101011110010010101010101000100111e-5 $ \\ 
%% &&&\\ \hline &&& \\ 
%% 9 &$ -7 $ & $\frac{8590943574074681}{4503599627370496}$ & $ .11110100001010110101011101100000001001010010100111000e-6 $ \\ 
%% &&&\\ \hline &&& \\ 
%% 10 &$ -7 $ & $\frac{-6188699963576117}{4503599627370496}$ & $ -.10101111111001001011011111110111100110100111100110101e-6 $ \\ 
%% &&&\\ \hline &&& \\ 
%% 11 &$ -7 $ & $\frac{5702144446736261}{4503599627370496}$ & $ .10100010000100001000111110110001101001010011110000100e-6 $ \\ 
%% &&&\\ \hline &&& \\ 
%% 12 &$ -8 $ & $\frac{-424997866135913}{281474976710656}$ & $ -.11000001010001000100001101101011101100001011010001111e-7 $ \\ 
%% &&&\\ \hline &&& \\ 
%% 13 &$ -3 $ & $\frac{-5757341135212823}{4503599627370496}$ & $ -.10100011101000100010101110001001101110001000100010111e-2 $ \\ 
%% &&&\\ \hline 
%% \end{tabular}
%% \end{table}


%% \begin{table}
%% \caption{Polynomial P 7} 
%% \begin{tabular}{|c|c|c|c|}
%% \hline &&& \\
%% coeff n�& exponent & mantissa & binary number \\ 
%% &&&\\ \hline &&& \\
%% 0 HI &$ -2 $ & $\frac{2868376209600353}{2251799813685248}$ & $ .10100011000011000101111000010000111000101111011000010e-1 $ \\ 
%% &&&\\ \hline &&& \\ 
%% 0 LO &$ -56 $ & $\frac{8799248525173221}{4503599627370496}$ & $ .11111010000101101111010110010101000100010000111100100e-55 $ \\ 
%% &&&\\ \hline &&& \\ 
%% 1 HI &$ -1 $ & $\frac{3275345183542179}{2251799813685248}$ & $ .10111010001011101000101110100010111010001011101000110 $ \\ 
%% &&&\\ \hline &&& \\ 
%% 1 LO &$ -56 $ & $\frac{-6555200597316529}{4503599627370496}$ & $ -.10111010010011110101110010011000010001101001110110001e-55 $ \\ 
%% &&&\\ \hline &&& \\ 
%% 2 &$ -2 $ & $\frac{-297758653049289}{281474976710656}$ & $ -.10000111011001111010101101011111001101001110010010000e-1 $ \\ 
%% &&&\\ \hline &&& \\ 
%% 3 &$ -3 $ & $\frac{2309885308503577}{2251799813685248}$ & $ .10000011010011010100000101010100100100000110000110001e-2 $ \\ 
%% &&&\\ \hline &&& \\ 
%% 4 &$ -4 $ & $\frac{-5039749764007811}{4503599627370496}$ & $ -.10001111001111010000000101110011100001100011110000011e-3 $ \\ 
%% &&&\\ \hline &&& \\ 
%% 5 &$ -5 $ & $\frac{1466109022249467}{1125899906842624}$ & $ .10100110101011010110001101101111001001000011111101100e-4 $ \\ 
%% &&&\\ \hline &&& \\ 
%% 6 &$ -6 $ & $\frac{-7108407380574949}{4503599627370496}$ & $ -.11001010000010000111100010000110101101000001011100101e-5 $ \\ 
%% &&&\\ \hline &&& \\ 
%% 7 &$ -7 $ & $\frac{8862430084207537}{4503599627370496}$ & $ .11111011111000101010101001101000001000001001110110001e-6 $ \\ 
%% &&&\\ \hline &&& \\ 
%% 8 &$ -7 $ & $\frac{-5639728263573257}{4503599627370496}$ & $ -.10100000010010100110110001111110001111010111100001000e-6 $ \\ 
%% &&&\\ \hline &&& \\ 
%% 9 &$ -8 $ & $\frac{3645840892850449}{2251799813685248}$ & $ .11001111001111011111011110110101001101110001000100010e-7 $ \\ 
%% &&&\\ \hline &&& \\ 
%% 10 &$ -8 $ & $\frac{-2386362174521563}{2251799813685248}$ & $ -.10000111101001100010001111011011011111000000110110110e-7 $ \\ 
%% &&&\\ \hline &&& \\ 
%% 11 &$ -9 $ & $\frac{3173561151701001}{2251799813685248}$ & $ .10110100011001010110010010010001111011000000000010010e-8 $ \\ 
%% &&&\\ \hline &&& \\ 
%% 12 &$ -10 $ & $\frac{-8467205555863361}{4503599627370496}$ & $ -.11110000101001110000011101011011001011011111101000000e-9 $ \\ 
%% &&&\\ \hline &&& \\ 
%% 13 & $ 0 $ & $ 0 $ & $ 0 $ \\ 
%% &&&\\ \hline 
%% \end{tabular}
%% \end{table}

%% Erreurs approx

%% P0:
%% -63.0335561543503153519561060236275821115193176282828598936000091969791396082118533396$

%% P1:
%% -62.1442297174139549747120885846224119378071897891851912972827117256112491790838501306$

%% P2:
%% -61.2333262205549290003996733087260577169790459154067176452546458667687388357147177354$

%% P3:
%% -60.4190986335205152072260108229699448188586300049015401437646252096816380253547811958$

%% P4:
%% -62.9461911106917443178043861037065674873169791834709346433334048487967658174065656673$

%% P5:
%% -60.2165945189274221232763599838279416602051132104701245665550547127176437549725373247$

%% P6:
%% -61.6071069727527318902661883152800182616307327795335555694099125712238629376830307421$

%% P7:
%% -63.0374421600999012862137048917495991411594406510710431721052393396795383773111066083$



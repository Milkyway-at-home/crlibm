This chapter is contributed by C. Daramy-Loirat, D. Defour
and F. de~Dinechin.  

\section*{Introduction}
This chapter describes the implementations of sine, cosine and
tangent, as they share much of their code. The proof sketch below is
supported by the Maple script \texttt{maple/trigo.mpl} of the \crlibm\
distribution, which implements the computations of error bounds and
validity bounds for the various algorithmic paths described here.

\section{Overview of the algorithms}

\subsection{Exceptional cases}

The three trigonometric functions return NaN for infinite and NaN
arguments, and are defined otherwise. $\tan$ may not return
infinities: an argument based on continued fractions to find the worst
cases for range reduction may also be used to show that the sine of a
floating-point number is always larger than $2^{-150}$, and therefore
never flushes to zero (see \ref{Muller97} p. 151 and following).

For very small arguments,
\begin{itemize}
\item $\sin(x) = x-x^3/6 + O(x^5) = x(1-x^2/6) + O(x^5)$ where
  $O(x^5)$ has the sign of $x$. Therefore $\sin(x)$ is rounded to $x$
  for all the rounding  modes if $|x|<2^{-26}$.
\item $\cos(x) = 1-x^2/2 + O(x^4)$ where $O(x^4)$ is positive.
  Therefore $\cos(x)$ is rounded to $1$ in RN and RU mode if
  $x<\sqrt{2^{-53}}$. In RD and RZ modes, we have $\cos(0)=1$ and
  $\cos(x)=1-2^{-53}$ for $|x|<2^{-26}$.
\item $\tan(x) = x+x^3/3 + O(x^5) = x(1+x^2/3) + O(x^5)$ where
  $O(x^5)$ has the sign of $x$. Therefore $\tan(x)$ is rounded to $x$
  for all the rounding  modes if $|x|<2^{-26}$.
\end{itemize}


\subsection{Argument reduction}

Most implementations of the trigonometric functions have two steps of argument reduction: 
\begin{itemize}
\item first the input number $x$ is reduced to $y\in
  [0;\frac{\pi}{2}]$, with reconstruction using periodicity and
  symmetry properties,
\item then the reduced argument is further broken down as $y=a+z$,
  with reconstruction using the formula for $\sin(a+z)$ and
  $\cos(a+z)$, using tabulated values of $\sin(a)$ and
  $\cos(a)$
\end{itemize}

We chose to implement argument reduction in one step only, which
computes an integer $k$ and a reduced argument y such that

\begin{equation}
  x = k\frac{\pi}{256} + y\label{eq:trigoargred}
\end{equation}
where $k$ is an integer and  $ |y| \leq \frac{\pi}{512}$.
This step computes $y$ as a double-double: $y\approx y_h+y_l$. 

In the following we note $a=k\pi/256$. 

Then we read off a table 

$$sa_h+sa_l \approx sin(a)$$
$$ca_h+ca_l \approx cos(a)$$

Only 64 quadruples $(sa_h,sa_l,ca_h,ca_l)$ are tabulated (amounting to
$64\times 8 \times 4 = 2048$ bytes), the rest is obtained by
periodicity and symmetry, implemented as masks and integer operations
on the integer $k$. For instance,  $a \mod 2\pi$ is implemented by $k \mod 512$,
$\pi/2-a$ is implemented as $128-k$, etc.



Then we use the reconstruction steps:

\begin{equation}        
  \sin(x) = \sin(a + y) =  \cos(a) \sin(y) +  \sin(a) \cos(y) 
  \label{eq:sinapy}
\end{equation}

\begin{equation}
  \cos(x) = \cos(a + y) = \cos(a) \cos(y) -  \sin(a) \sin(y) 
  \label{eq:cosapy}
\end{equation}

\begin{equation} 
  tan(x) = \frac{\sin(x)}{\cos(x)} = \frac{\cos(a) \sin(y) +  \sin(a) \cos(y)}{\cos(a) \cos(y) -  \sin(a) \sin(y)}
  \label{eq:tanapy}
\end{equation}


\subsection{Polynomial evaluation}


To implement the previous equations, $\cos(y)$ and $\sin(y)$ are
computed as unevaluated $1+t_c$ and $(y_h+y_l)(1+t_s)$ respectively,
where $t_c$ and $t_s$ are doubles computed using a polynomial
approximation of small degree:

\begin{itemize}
\item $t_s = y^2*(s_3 + y^2*(s_5 + y^2*s_7)))$ with $s3$, $s5$ and
$s7$ the Taylor coefficients.
\item $t_c = y^2*(c_2 + y^2*(c_4 + y^2*c_6))$ with $c2$, $c4$ and $c6$ the
Taylor coefficients (or a more accurate minimax approximation).
\end{itemize}



\subsection{Reconstruction}

\subsubsection{Sine}
According to equation (\ref{eq:sinapy}), we have to compute: 
 \begin{eqnarray*}
  \sin(a+y) &=& \sin(a) \cos(y)  + \cos(a)\sin(y)  \\
  & \approx& (sa_h+sa_l)(1+t_c) + (ca_h+ca_l)(y_h+y_l)(1+t_s)
\end{eqnarray*}


Figure~\ref{fig:sine-reconstruction} shows the worst-case respective
orders of magnitude of the terms of this sum. The terms completely to the
right of the $\epsilon$ bar will be neglected, and a bound on the
error thus entailed is computed in the following. Note that the term
$ca_hy_h$ has to be computed exactly by a Mul12.

Finally the reconstruction consists of adding together the lower-order
terms in increasing order of magnitude, and computing the
double-double result by an Add12.

\begin{figure}[htbp]    
  \begin{center}
    \small
    \setlength{\unitlength}{3ex}
      \framebox{
        \begin{picture}(22,9)(-3,-4.2)
          \put(9.5,4){\line(0,-1){8}}  \put(9,4){$\epsilon$}
          
          \put(4,3.2){$sa_h$} \put(0.05,3){\framebox(7.9,0.7){}}
          \put(12,3.2){$sa_l$}  \put(8.05,3){\framebox(7.9,0.7){}}
          
          \put(6,2.2){$sa_ht_c$} \put(2.05,2){\framebox(7.9,0.7){}}
          \put(14,2.2){$sa_lt_c$}  \put(10.05,2){\framebox(7.9,0.7){}}

          \put(4.5,1.2){$ca_hy_h$} \put(0.55,1){\framebox(7.9,0.7){}}
          \put(12.5,1.2){$ca_hy_h$}  \put(8.55,1){\framebox(7.9,0.7){}}

          \put(11.5,0.2){$ca_hy_l $}  \put(7.55,0){\framebox(7.9,0.7){}}
          \put(11.5,-0.8){$ca_ly_h $}  \put(7.55,-1){\framebox(7.9,0.7){}}

         \put(6.5,-1.8){$ca_hy_ht_s$} \put(2.55,-2){\framebox(7.9,0.7){}}
         \put(13.5,-2.8){$ca_hy_lt_s $}  \put(9.55,-3){\framebox(7.9,0.7){}}
         \put(13.5,-3.8){$ca_ly_ht_s $}  \put(9.55,-4){\framebox(7.9,0.7){}}
        
       \end{picture}
     }
   \end{center}
   \caption{The sine reconstruction}
   \label{fig:sine-reconstruction}
 \end{figure}
 
 

\subsubsection{Cosine}
According to equation (\ref{eq:cosapy}), we have to compute in double-double precision:
 \begin{eqnarray*}
  \cos(a+y) &=& \cos(a) \cos(y)  - \sin(a)\sin(y)  \\
  & \approx& (ca_h+ca_l)(1+t_c) - (sa_h+sa_l)(y_h+y_l)(1+t_s)
\end{eqnarray*}

This is similar to the case of the sine, and the respective orders of
magnitude are given by Figure~\ref{fig:sine-reconstruction}.

\begin{figure}[htbp]
  \begin{center}
    \small \setlength{\unitlength}{3ex} \framebox{
      \begin{picture}(22,9)(-3,-4.2)
        \put(9.5,4){\line(0,-1){8}}
        \put(9,4){$\epsilon$}
  
        \put(4,3.2){$ca_h$} \put(0.05,3){\framebox(7.9,0.7){}}
        \put(12,3.2){$ca_l$}  \put(8.05,3){\framebox(7.9,0.7){}}

        \put(6,2.2){$ca_ht_c$} \put(2.05,2){\framebox(7.9,0.7){}}
        \put(14,2.2){$ca_lt_c$}  \put(10.05,2){\framebox(7.9,0.7){}}

        \put(4.5,1.2){$-sa_hy_h$} \put(0.55,1){\framebox(7.9,0.7){}}
        \put(12.5,1.2){$-sa_hy_h$}  \put(8.55,1){\framebox(7.9,0.7){}}

        \put(11.5,0.2){$-sa_hy_l $}  \put(7.55,0){\framebox(7.9,0.7){}}
        \put(11.5,-0.8){$-sa_ly_h $}  \put(7.55,-1){\framebox(7.9,0.7){}}

        \put(6.5,-1.8){$-sa_hy_ht_s$} \put(2.55,-2){\framebox(7.9,0.7){}}
        \put(13.5,-2.8){$-sa_hy_lt_s $}  \put(9.55,-3){\framebox(7.9,0.7){}}
        \put(13.5,-3.8){$-sa_ly_ht_s $}  \put(9.55,-4){\framebox(7.9,0.7){}}
 
        \end{picture}
      }
    \end{center}\centering
    
    \caption{The cosine reconstruction}
    \label{fig:cosine-reconstruction}
  \end{figure}


\subsubsection{Tangent}

The tangent is obtained by the division of the sine by the cosine.
The procedure "Div22" guarantees a $2^{-104}$ precision result, wich is surely enough to have a precise final result.


\subsection{Precision of this scheme}

As we have $|y|<\pi/512<2^{-7}$, this scheme computes these functions accurately to roughly $53+12$
bits, so these first steps are  very accurate.
 



\section{Details of argument reduction}


We have 4 possible range reductions, depending on the magnitude of the input number:

\begin{itemize}
\item Cody and Waite with 2 constants (the fastest),
\item Cody and Waite with 3 constants (almost as fast),
\item Cody and Waite with 3 constants in double-double and $k$ a
  64-bit int,
\item Payne and Hanek, implemented in SCS (the slowest).
\end{itemize}
Each of these range reductions except Payne and Hanek is valid for $x$
smaller than some bound. The computation of these bounds is detailed
below.

Such a range reduction may cancel up to 62 bits according to a program
by Kahan/Douglas available in Muller's book \cite{Muller97} and
implemented as function \texttt{WorstCaseForAdditiveRangeReduction} in
\texttt{maple/common-procedures.mpl}.  However this is not a concern
unless x is close to a multiple of $\pi/2$ (that is, $k \mod 128=0$): in
the general case the reconstruction will add some tabulated non-zero
value, so the error to consider in the range reduction is the absolute
error.  Only in the cases when $k \mod 128=0$ do we need to have 62
extra bits to compute with. This is ensured by using a slower, more
accurate range reduction. As a compensation, in this case when $k \mod
128=0$, there is no table to read and no reconstruction to perform: a
simple polynomial approximation to the function suffices.






\section{Detailed examination of the sine}

The sine begins with casting the high part of the absolute value of
the input number into a 32-bit integer, to enable faster comparisons.
However we have to be aware that we lost the lower part of \texttt{x},
which had a value up to $(2^{31}-1)\ulp(x)$. This is taken care of in
the procedure that converts a Maple high-precision number into an
integer to which \texttt{x} will be compared (procedure
\texttt{outputHighPart} in \texttt{trigo.mpl}).

\begin{lstlisting}[caption={Casting to an int for faster comparisons},firstnumber=1]
  db_number x_split;
  x_split.d=x;
  absxhi = x_split.i[HI_ENDIAN] & 0x7fffffff;
\end{lstlisting}


\subsection{Exceptional cases in RN mode}
\begin{lstlisting}[caption={Exceptional cases for sine RN},firstnumber=1]
  /* SPECIAL CASES: x=(Nan, Inf) sin(x)=Nan */
  if (absxhi>=0x7ff00000) return x-x;    
   
  else if (absxhi < XMAX_SIN_CASE2){
    /* CASE 1 : x small enough sin(x)=x */
    if (absxhi <XMAX_RETURN_X_FOR_SIN)
      return x;
\end{lstlisting}

\subsection{Exceptional cases in RU mode}
\begin{lstlisting}[caption={Exceptional cases for sine RU},firstnumber=1]
  /* SPECIAL CASES: x=(Nan, Inf) sin(x)=Nan */
  if (absxhi>=0x7ff00000) return x-x;    
  
  if (absxhi < XMAX_SIN_CASE2){

    /* CASE 1 : x small enough, return x suitably rounded */
    if (absxhi <XMAX_RETURN_X_FOR_SIN) {
      if(x>=0.)
	return x;
      else {
	x_split.l --;
	return x_split.d;
      }
    }
\end{lstlisting}
\subsection{Exceptional cases in RD mode}
\begin{lstlisting}[caption={Exceptional cases for sine RD},firstnumber=1]
  /* SPECIAL CASES: x=(Nan, Inf) sin(x)=Nan */
  if (absxhi>=0x7ff00000) return x-x;    
  
  if (absxhi < XMAX_SIN_CASE2){

    /* CASE 1 : x small enough, return x suitably rounded */
    if (absxhi <XMAX_RETURN_X_FOR_SIN) {
      if(x<=0.)
	return x;
      else {
	x_split.l --;
	return x_split.d;
      }
    }
\end{lstlisting}
\subsection{Exceptional cases in RZ mode}
\begin{lstlisting}[caption={Exceptional cases for sine RZ},firstnumber=1]
  /* SPECIAL CASES: x=(Nan, Inf) sin(x)=Nan */
  if (absxhi>=0x7ff00000) return x-x;    
  
  if (absxhi < XMAX_SIN_CASE2){

    /* CASE 1 : x small enough, return x suitably rounded */
    if (absxhi <XMAX_RETURN_X_FOR_SIN) {
      x_split.l --;
      return x_split.d;
    }
\end{lstlisting}
\subsection{Fast approximation of sine for small arguments \label{sec:trigo:fastsine}}

\begin{lstlisting}[caption={Sine, case 2},firstnumber=1]
    xx = x*x;
    ts = xx * (s3.d + xx*(s5.d + xx*s7.d ));
    Add12(sh,sl, x, x*ts);
\end{lstlisting}

Here we have had no argument reduction, therefore \texttt{x} is exact.
We need to compute the relative error of $\mathtt{sh}+\mathtt{sl}$
with respect to $\sin(\mathtt{x})$. As $\mathtt{sh}+\mathtt{sl}$ is
the result of an (exact) \texttt{Add12}, the error is:

\begin{equation}
  \epsilon_{\mathrm{sinCase2}} = \frac{\mathtt{x}\otimes \mathtt{ts} + \mathtt{x}}{\sin(\mathtt{x})} -1 = \frac{\mathtt{x}\times\mathtt{ts}(1+\epsilon_{-53}) + \mathtt{x}}{\sin(\mathtt{x})} -1
\label{eq:SinCase2Total}
\end{equation}
The polynomial used to compute \texttt{ts}
approximates $\frac{\sin(x)-x}{x}$: 
$$
P_{\mathtt{ts}}(x) = \mathtt{s3}.x^2 + \mathtt{s5}.x^4 + \mathtt{s7}.x^6
= \frac{\sin(x)-x}{x}(1+\epsilon_{\mathrm{approxts}})
$$

We compute a bound on this error in Maple as 
$$\maxeps_{\mathrm{approxts}} = \left\Vert \frac{xP_{\mathtt{ts}}(x)}{\sin(x)-x} -1 \right\Vert_{\infty}[-x_{\mathrm{max}}...x_{\mathrm{max}}]$$

We also compute $\epsilon_{\mathrm{hornerts}}$, the relative error due
to rounding in the Horner evaluation thanks to the
\texttt{compute\_horner\_rounding\_error} procedure. For this we need
the relative error carried by \texttt{xx}, which is only due to the
rounding error in the multiplication since \texttt{x} is exact:
$$\mathtt{xx}=\mathtt{x}^2(1+\epsilon_{-53})$$

We therefore have:

$$\mathtt{ts} = P_{\mathtt{ts}}(\mathtt{x})(1+\epsilon_{\mathrm{hornerts}}) = \frac{\sin(\mathtt{x})-\mathtt{x}}{\mathtt{x}}(1+\epsilon_{\mathrm{approxts}})(1+\epsilon_{\mathrm{hornerts}})$$

Reporting this in (\ref{eq:SinCase2Total}), we get 
\begin{equation*}
  \epsilon_{\mathrm{sinCase2}} = \frac{(\sin(\mathtt{x})-\mathtt{x})(1+\epsilon_{\mathrm{approxts}})(1+\epsilon_{\mathrm{hornerts}})(1+\epsilon_{-53}) + \mathtt{x}}{\sin(\mathtt{x})} -1
\end{equation*}
or,
\begin{equation*}
  \epsilon_{\mathrm{sinCase2}} =  \frac{\sin(\mathtt{x})-\mathtt{x}}{\sin(\mathtt{x})}\left((1+\epsilon_{\mathrm{approxts}})(1+\epsilon_{\mathrm{hornerts}})(1+\epsilon_{-53})\ -\ 1\right)
\end{equation*}

Finally
\begin{equation}
  \maxeps_{\mathrm{sinCase2}} =  \left\Vert\frac{\sin(\mathtt{x})-\mathtt{x}}{\sin(\mathtt{x})}\right\Vert_{\infty}((1+\maxeps_{\mathrm{approxts}})(1+\maxeps_{\mathrm{hornerts}})(1+\maxeps_{-53})\ -\ 1)
  \label{eq:SinCase2Total2}
\end{equation}
 




\section{Detailed examination of auxilliary functions}

\subsection{Errors due to argument reduction}
Upon entering one of \texttt{do\_sin\_k\_notzero},
\texttt{do\_cos\_k\_zero}, \texttt{do\_cos\_k\_notzero}, we have in
$y_h+y_l$ an approximation to the ideal reduced value
$\hat{y}=x-k\pi/256$ with a relative accuracy $\epsilon_{argred}$:

\begin{equation}
  y_h+y_l = (x-k\pi/256)(1+\epsilon_{argred}) = \hat{y}(1+\epsilon_{argred})
  \label{eq:sinargrederror1}
\end{equation}
whith, depending on the quadrant, $\sin(\hat{y} = \pm\sin(x)$ or
$\sin(\hat{y} = \pm\cos(x)$: the important idea is that there is no
error in this ideal argument reduction. In the following we will
assume we are in the case $\sin(\hat{y} = \sin(x)$, (the proof is
identical in the other cases) therfore the relative error that we need
to compute is
\begin{equation}
  \epsilon_{\mathrm{sinkzero}} = \frac{(\mathtt{*psh} + \mathtt{*psl})}{\sin(\mathtt{x})} -1 = \frac{(\mathtt{*psh} + \mathtt{*psl})}{\sin(\hat{y})} -1
\end{equation}

We also have a common upper bound $y_{\max}$ on $|\hat{y}|$,
$|\mathtt{yh}+\mathtt{yl}|$ and $|\mathtt{yh}|$, which also takes into
account the fact that we used a machine rounding to get $k$, so the
reduced value may exceed $\pi/512$:
\begin{equation}
  y_{\max} = \frac{pi}{512}(1+\epsilon_{\mathrm{argred2}})
  \label{eq:ymaxsink0}  
\end{equation}
where $\epsilon_{\mathrm{argred2}}$ should be computed somewhere else (TODO).


\subsection{\texttt{do\_sin\_k\_zero}}


\begin{lstlisting}[caption={do\_sin\_k\_zero},firstnumber=1]
#define do_sin_k_zero(psh,psl, yh,  yl)            \
do{                                                \
  double yh2 ;	              			   \
  yh2 = yh*yh;					   \
  ts = yh2 * (s3.d + yh2*(s5.d + yh2*s7.d));	   \
  /* (1+ts)*(yh+yl) is an approx to sin(yh+yl) */  \
  /* Now compute (1+ts)*(yh+yl) */                 \
  Add12(*psh,*psl,   yh, yl+ts*yh);	           \
} while(0)						   
\end{lstlisting}

One can see that we almost have the same code as in
\ref{sec:trigo:fastsine}. The difference is that we have as input
a double-double $\mathtt{yh}+\mathtt{yl}$, which is itself an inexact term. 

At Line 4, the error of neglecting $y_l$ and the rounding error in the
multiplication each amount to half an ulp:
  $\mathtt{yh2}=\mathtt{yh}^2(1+\epsilon_{-53})$, 
 with $\mathtt{yh} = (\mathtt{yh}+\mathtt{yl})(1+\epsilon_{-53}) = \hat{y}(1+\epsilon_{\mathrm{argred}})(1+\epsilon_{-53})$

Therefore
\begin{equation}
  \mathtt{yh2}=\hat{y}^2(1+\epsilon_{\mathtt{yh2}})
\end{equation}

with
\begin{equation}
  \maxeps_{\mathtt{yh2}} = (1+\maxeps_{\mathrm{argred}})^2(1+\maxeps_{-53})^3 - 1
\end{equation}

Line 5 is a standard Horner evaluation. Its approximation error is defined by: 
$$
P_{\mathtt{ts}}(\hat{y}) = \frac{\sin(\hat{y})-\hat{y}}{\hat{y}}(1+\epsilon_{\mathrm{approxts}})
$$

This error is computed in Maple as in \ref{sec:trigo:fastsine}, only the interval changes:
$$\maxeps_{\mathrm{approxts}} = \left\Vert \frac{xP_{\mathtt{ts}}(x)}{\sin(x)-x} -1 \right\Vert_{\infty}$$

We also compute $\maxeps_{\mathrm{hornerts}}$, the bond on the relative error due
to rounding in the Horner evaluation thanks to the
\texttt{compute\_horner\_rounding\_error} procedure. This time, this procedure 
takes into account the relative error carried by \texttt{yh2}, which is
$\maxeps_{\mathtt{yh2}}$ computed above.
We thus get the total relative error on \texttt{ts}:

\begin{equation}
  \mathtt{ts} = P_{\mathtt{ts}}(\hat{y})(1+\epsilon_{\mathrm{hornerts}}) = \frac{\sin(\hat{y})-\hat{y}}{\hat{y}}(1+\epsilon_{\mathrm{approxts}})(1+\epsilon_{\mathrm{hornerts}})
  \label{eq:sink0ts}
\end{equation}

The final \texttt{Add12} is exact. Therefore the overall relative error is:

\begin{eqnarray*}
  \epsilon_{\mathrm{sinkzero}} 
  &=& \frac{((\mathtt{yh}\otimes \mathtt{ts}) \oplus \mathtt{yl}) + \mathtt{yh}}{\sin(\hat{y})} -1 \\
  &=& \frac{(\mathtt{yh}\otimes\mathtt{ts} + \mathtt{yl})(1+\epsilon_{-53}) + \mathtt{yh}}{\sin(\hat{y})} -1\\
  &=& \frac{\mathtt{yh}\otimes\mathtt{ts} + \mathtt{yl} + \mathtt{yh}    \ +\  (\mathtt{yh}\otimes\mathtt{ts} + \mathtt{yl}).\epsilon_{-53}}{\sin(\hat{y})} -1\\
\end{eqnarray*}

Define 
\begin{equation}
  \delta_{\mathrm{addsin}} = \max((\mathtt{yh}\otimes\mathtt{ts} + \mathtt{yl}).\epsilon_{-53})
\label{eq:addsin}
\end{equation}
which is computed out of the  the maximum values taken by \texttt{ts}, \texttt{yh} and \texttt{yl}. Then we have
\begin{eqnarray*}
  \epsilon_{\mathrm{sinkzero}} 
  &=& \frac{(\mathtt{yh} + \mathtt{yl})\mathtt{ts}(1+\epsilon_{-53})^2 + \mathtt{yl} + \mathtt{yh}    \ +\  \delta_{\mathrm{addsin}} }{\sin(\hat{y})} -1\\
\end{eqnarray*}

Using (\ref{eq:sinargrederror1}) and (\ref{eq:sink0ts}) we get:
\begin{eqnarray*}
  \epsilon_{\mathrm{sinkzero}} 
  &=& \frac{\hat{y}(1+\epsilon_{\mathrm{argred}})\times\frac{\sin(\hat{y})-\hat{y}}{\hat{y}}(1+\epsilon_{\mathrm{approxts}})(1+\epsilon_{\mathrm{hornerts}})(1+\epsilon_{-53})^2 + \mathtt{yl} + \mathtt{yh}    \ +\  \delta_{\mathrm{addsin}} }{\sin(\hat{y})} -1\\
\end{eqnarray*}

To lighten notations, let us define 
\begin{equation}
 \maxeps_{\mathrm{sin1}} = (1+\maxeps_{\mathrm{approxts}})(1+\maxeps_{\mathrm{hornerts}})(1+\maxeps_{-53})^2 \ -\ 1
  \label{eq:epssin1}
\end{equation}

We get
\begin{eqnarray*}
  \epsilon_{\mathrm{sinkzero}} 
  &=& \frac{(\sin(\hat{y})-\hat{y})(1+\epsilon_{\mathrm{sin1}}) + \hat{y}(1+\epsilon_{\mathrm{argred}})    \ +\   \delta_{\mathrm{addsin}} - \sin(\hat{y})}{\sin(\hat{y})}\\
  &=& \frac{(\sin(\hat{y})-\hat{y}).\epsilon_{\mathrm{sin1}} + \hat{y}.\epsilon_{\mathrm{argred}}    \ +\ \delta_{\mathrm{addsin}}}{\sin(\hat{y})}\\
\label{eq:sinkzero}
\end{eqnarray*}

Again we may compute the value of $\maxeps_{\mathrm{sinkzero}}$ as an infinite norm under Maple. Or, we may compute majorations.

 



\subsection{\texttt{do\_sin\_k\_notzero}}


\section{Overview of the algorithm}
The arcsine $\arcsin\left( x \right) = \sum\limits_{i=0}^{\infty}
\frac{\left( 2i - 1 \right)!!}{\left(2i+1\right) \cdot \left( 2i
\right)!!} \cdot x^{2i+1}$ is defined on the domain $x \in \left[ -1;
1 \right]$. It is a odd function: $\arcsin\left(-x\right) =
-\arcsin\left( x \right)$. Its value in $0$ is $\arcsin\left( 0
\right) = 0$. Its derivative tends to infinity when $x$ tends to $1$:
$\lim\limits_{x \rightarrow 1} \left( \frac{d}{dx} \arcsin
\right)\left(x\right) = \infty$; the function's value in $1$ is
nevertheless finite: $\arcsin\left( 1 \right) = \frac{\pi}{2}$.  There
is no simple additive or multiplicative decomposition of this
function.

A correctly rounded implementation of $\arcsin$ must provide an
accuracy of at least $126$ bits for $\left \vert x \right \vert \leq
2^{-18}$ and of at least $118$ bits for the rest of the definition
domain in the accurate phase \cite{DinDefLau2004LIP}.

The algorithm chosen principally consists of a piecewise polynomial
approximation either of the function itself or of a asymptotic
development of the function. More precisely, the following is done:
\begin{itemize}
\item Special cases, such as $\left \vert x \right \vert > 1$, $x = \pm \infty$, $x = \nan$, are handled.
\item The sign of the argument $x$ is stripped off because
$\arcsin\left(x\right) = \sgn\left(x\right) \cdot \arcsin\left( \left
\vert x \right \vert \right)$. We will suppose in the following, that
$x$ stands for a positive argument, $x \geq 0$.
\item The argument is classified in one of $10$ subdomains of $\left[
0; 1 \right]$.  This means an integer $i \in \left[0\dots9\right]$ is
computed such that $x \in I_i$ where
$$\bigcup\limits_{i=0}^9 I_i = \left[0; 1 \right]$$
$$i \not= j \Rightarrow I_i \cap I_j = \emptyset$$
$$i < j \Rightarrow \forall x_i \in I_i, x_j \in I_j \mbox{ . } x_i < x_j$$
\item If $i = 0$, $\arcsin$ is directly approximated as 
$$\arcsin\left( x \right) \approx x + x^3 \cdot p_0\left( x^2 \right)$$
\item If $1 \leq i \leq 8$, an interval midpoint value $m_i
\approx \frac{\inf I_i + \sup I_i}{2}$ is read in a table.  The function is
then approximated as
$$\arcsin\left( x \right) \approx \arcsin\left( m_i \right) +  \left(x - m_i \right) \cdot p_i\left( x - m_i \right)$$
\item If $i = 9$, $\arcsin$ is approximated as 
$$\arcsin\left( x \right) \approx p_9\left( 1 - x \right) \cdot \sqrt{2 - 2\cdot x} + \frac{\pi}{2}$$
\end{itemize}
The polynomials $p_i$, $1 \leq i \leq 8$, for the middle intervals are
all of the same degree and, loaded from a table, can be evaluated in
the same computation path. The polynomials $p_0$ and $p_9$ are of
different degree. So there are three distinct paths in the code. In
the following, they are referred to as the low, middle and high path.

Concerning the quick and accurate phase of the implementation, it must
mentioned that code and cache size considerations do not allow for
using different coefficient tables for the quick phase polynomials
$p_{i\mbox{\tiny ~quick}}$ as for the accurate phase polynomials
$p_{i\mbox{\tiny ~accurate}}$ that are obviously longer must contain
more accurately stored coefficients. So the quick phase polynomials are
simply the accurate phase polynomials truncated to some degree and simplified 
by omitting low significance components of the coefficients.

The double precision midpoint values $m_i \in \F$ for the middle path
intervals $I_i$, $1 \leq i \leq 8$, are chosen such that a
double-double approximation $asinm_{i\hi} + asinm_{i\lo}$ of
$\arcsin\left( m_i \right)$ is accurate to at least $121$ bits. This
allows for saving up memory in the tables used.

The intervals $I_i$ are not uniformly distributed. This has the
disadvantage that the computation of $i$ cannot be done by simple
bitmasks on the arcsine's argument $x$ but that a dichotomy must be
performed. On the other hand, this is the only way of using
polynomials for the same degree for all middle intervals without
wasting accuracy for the lower ones. In fact, the derivative of
$\arcsin$ grows over-polynomially, which means that the polynomial
degrees must increase for equally sized reduced arguments in order to
achieve the same approximation error. 

The decomposition of the domain $\left[0;1\right]$ into the $I_i$s is
relatively ad-hoc. The given implementation uses:
$$I_i = \left[s_i;s_{i+1}\right]$$
with
\begin{eqnarray*}
s_0 & = & 0 \\
s_1 & = & 0.184999942779541015625  \\
s_2 & = & 0.2997171878814697265625  \\
s_3 & = & 0.40296268463134765625  \\
s_4 & = & 0.4932067394256591796875  \\
s_5 & = & 0.5696849822998046875  \\
s_6 & = & 0.639662265777587890625  \\
s_7 & = & 0.696256160736083984375 \\
s_8 & = & 0.741760730743408203125  \\
s_9 & = & 0.77999973297119140625  \\
s_{10} & = & 1 
\end{eqnarray*}
One remarks that in order to simplify the dichotomy for computing $i$,
the bounds of the intervals are chosen all such that the low order
word of the double precision numbers they are stored in are $0$. See
\ref{sec:asinargred}, page \pageref{sec:asinargred} for more precise
considerations on that subject.

Let be
$$z_i = \left \lbrace \begin{array}{ll} x & \mbox{ if } i = 0 \\
x - m_i & \mbox{ if } 1 \leq i \leq 8 \\
1 - x & \mbox{ if } i = 9 \end{array} \right.$$
This value $z_i$ is the argument to the polynomial $p_i$. With the given interval bounds, it is bounded by 

$$\left \vert z_1 \right \vert  \leq  2^{-2.434403}$$
$$\left \vert z_2 \right \vert  \leq  2^{-2.638419}$$
$$\left \vert z_3 \right \vert  \leq  2^{-2.817391}$$
$$\left \vert z_4 \right \vert  \leq  2^{-3.039782}$$
$$\left \vert z_5 \right \vert  \leq  2^{-3.208009}$$
$$\left \vert z_6 \right \vert  \leq  2^{-3.448784}$$
$$\left \vert z_7 \right \vert  \leq  2^{-3.760206}$$
$$\left \vert z_8 \right \vert  \leq  2^{-4.035253}$$
$$\left \vert z_9 \right \vert  \leq  2^{-2.064195}$$
$$\left \vert z_{10} \right \vert  \leq  2^{-2.184423}$$

The degrees of the polynomials $p_i$ and the number of double (D),
double-double (DD) and triple-double (TD) coefficients are listed below. Here the degree of the polynomial 
is the highest exponent of the monomial whose coefficient is not equal to $0$. We repeat that the 
$p_i$ are such that
$$\arcsin\left( x \right) \approx x + x^3 \cdot p_0\left( x^2 \right)$$
$$\arcsin\left( x \right) \approx \arcsin\left( m_i \right) + \left(x
- m_i \right) \cdot p_i\left( x - m_i \right), \mbox{~~} 1 \leq i \leq
8$$
$$\arcsin\left( x \right) \approx p_9\left( 1 - x \right) \cdot \sqrt{2 - 2\cdot x} + \frac{\pi}{2}$$

\begin{center}
\begin{tabular}{|l|l|l|l|l|l|l|l|}
\hline
$i$ & quick phase & accurate phase & D quick & DD quick & D accur. & DD accur. & TD accur. \\
\hline
\hline
$0$ & 8 & 17 & 5 & 4 & 7 & 6 & 5 \\ 
\hline
$1\dots8$ & 14 & 34 & 8 & 7 & 20 & 9 & 6 \\ 
\hline
$9$ & 17 & 28 & 10 & 8 & 12 & 9 & 8 \\ 
\hline
\end{tabular}
\end{center}
Remark that the listing does not account neither for the $8$
double-double values representing approximations to $\arcsin\left( m_i
\right)$ nor for the $8$ double precision numbers $m_i$.

\section{Interval discrimination and argument reduction}\label{sec:asinargred}

\section{Polynomial approximation and reconstruction}\label{sec:asinpolynomial}

\subsection{Quick phase polynomial approximation and reconstruction}\label{subsec:asinquickpolynomial}

\subsection{Accurate phase polynomial approximation and reconstruction}\label{subsec:asinaccpolynomial}

\section{Accuracy bounds}\label{sec:asinaccuracy}

\section{Timings}\label{sec:asintiming}

%%% Local Variables: 
%%% mode: latex
%%% TeX-master: "crlibm"
%%% End: 

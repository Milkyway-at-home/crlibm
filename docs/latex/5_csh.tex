This chapter is contributed by Matthieu Gallet under the supervision of
F. de~Dinechin.  


\section{Overview}

Like the algorithms for others elementary functions, we will compute
our hyperbolic cosine and sine in one or two steps.  The first one,
which is called 'fast step' is always executed and provides a
precision of about 63 bits.  This is a sufficient precision in most
cases, but sometimes more precision is needed, and then we enter the
second step using the SCS library. 



\subsection{Definition interval and exceptional cases}
The hyperbolic cosine and the hyperbolic sine are defined for all real
numbers, and then for all floating point numbers.  These functions are
divergent toward $+\infty$ and $-\infty$, so for $|x| > 710.47586007$,
$\cosh(x)$ and $\sinh(x)$ should return $+\infty$ or the greatest
representable number (depending ot the choosen rounding mode).

\begin{itemize}
\item If $x = NaN$ , then $\sin(x)$ and $\cosh(x)$ should return $NaN$
\item If $x = +\infty$ , then $\cosh(x)$ and $\sinh(x)$ should return $+\infty$. 
\item If $x = -\infty$ , then $\cosh(x)$ should return $+\infty$. 
\item If $x = -\infty$ , then $\sinh(x)$ should return $-\infty$. 
\end{itemize}

This is true in all rounding modes.

Concerning subnormals, $\cosh(x) \geq 1$, so $\cosh(x)$ can't return
subnormal numbers, even for subnormal inputs.  For small inputs ($|x|
\leq 2^{-26}$), we have $\sinh(x) = x$ with 80 bits of precision, so
we can return a result without any computation on subnormals.

\subsection{Relation between $\cosh(x)$, $\sinh(x)$ and $e^x$}

The hyperbolic sine and cosine are defined by 
 $$\sinh(x) = \frac{e^x -  e^{-x}}{2}$$ 
and
 $$\cosh(x) = \frac{e^x +  e^{-x}}{2}$$
respectively.

For large arguments, we will be able to neglect the smaller term. In
the first step, we will determine a threshold at which
$e^{-x}<2^{-65}e^x$. In the second step, the threshold will be such
that $e^{-x}<2^{-130}e^x$. 

Note that this has been used in the search for worst cases, too:
knowing that the worst case for exp requires $2^{-113}$ accuracy, and
the division by 2 being exact, we deduce that the worst cases for
$\sinh$ and $\cosh$ will be the same that for the exponential.
However, due to the division by 2, the domain of $\sinh$ and $\cosh$
is slightly larger than that of exp, so there is a little additional
search to do (precisely between 709.78 and 710.75). For the same
reason, this additional search gives worst cases for both $\sinh$ and
$\cosh$.


\subsection{Worst cases for correct rounding}
The search for the worst-case accuracy required for correct rounding
the hyperbolic sine and cosine and their inverses is completed. The
$\sinh$ and $\sinh^{-1}$ functions require a relative accuracy of
$2^{-126}$ in the worst case, while the $\cosh$ and $\cosh^{-1}$ require
$2^{-115}$. The second step will use the accurate exponential, and
will therefore be accurate to $2^{-150}$. This ensure coverage of all
the most difficult cases.



\section{Quick phase}

\subsection{Overview of the algorithm}

The algorithm consists of two argument reduction using classical
formulae of hyperbolic trigonometry, followed by a polynomial
evaluation using a Taylor polynom of degree $6$ (for $\cosh$) and $7$
(for $\sinh$).

These formulaes are:
\begin{itemize}
\item  $\sinh(x + y) = \sinh(x)  \cosh(y) + \sinh(y)  \cosh(x)$
\item  $\cosh(x + y) = \cosh(x)  \cosh(y) + \sinh(x)  \sinh(y)$
\item  $\cosh(k\ln(2)) = 2^{k-1} + 2^{-k-1}$
\item   $\sinh(k\ln(2)) = 2^{k-1} - 2^{-k-1}$
\end{itemize}



After having treated special cases ($NaN$, $+\infty$, $-\infty$), we
do a first range reduction to reduce the argument between
$\frac{-ln(2)}{2}$ and $\frac{ln(2)}{2}$.  So, we write $x = k\ln(2)
+ y$, where k is given by rounding to the nearest integer $x 
\frac{1}{ln(2)}$.  Now, $\frac{-ln(2)}{2} \leq y \leq
\frac{ln(2)}{2}$, but it is even too large to have a sufficient
precision during polynomial evaluation with small polynoms, and we do
a second range reduction, by writing $y = a + b$, where $a = index .
2^{-8}$ (index is an integer) and $|b| \leq 2^{-9}$.

Mathematically, we have: $$\sinh(x) = (2^{k-1} + 2^{-k-1})\sinh(y) +
(2^{k-1} - 2^{-k-1}) \cosh(y)$$
and $$\cosh(x) = (2^{k-1} +
2^{-k-1})\cosh(y) + (2^{k-1} - 2^{-k-1}) \sinh(y)$$
The second range
reduction allows to compute $\sinh(y)$ and $\cosh(y)$ as $\sinh(y) =
\sinh(a) \cosh(b) + \sinh(b) \cosh(a)$ and $\cosh(y) = \cosh(a) 
\cosh(b) + \sinh(a) \sinh(b)$. In the C code, we have $ch\_hi + ch\_lo
\approx \cosh(y)$ and $sh\_hi + sh\_lo \approx \sinh(y)$.

A quick computation shows that $-89 \leq index \leq 89$, and we can
pre-compute so few values of $\sinh(a)$ and $\cosh(a)$ and store them in
a table as double-doubles.


The constants $2^{k-1}$ and $2^{-k-1}$ are constructed by working
directly on their binary representation.


$\cosh(b)$ and $\sinh(b)$ are computed with Taylor polynoms. It's
well-known that $$\cosh(b) = \sum_{n \geq 0}{\frac{x^{2n}}{(2n)!}}$$
and $$\sinh(b) = \sum_{n \geq 0}{\frac{x^{2n+1}}{(2n+1)!}}$$
For our
needs, a degree $6$ polynom for cosh and a degree $7$ polynom for sinh
give enough accuracy.

We write $\cosh(b) = 1 + tcb$ and $\sinh(b) = b(1 + tsb)$, where
$$tcb = b^{2}  (\frac{1}{2} + b^{2}(\frac{1}{24} + b^{2} 
\frac{1}{720}))$$
$$tsb = b^{2} (\frac{1}{6} + b^{2} 
(\frac{1}{120} + b^{2}  \frac{1}{5040}))$$
We use the Horner
scheme  for the evaluation of the polynoms, with all the
coefficients being coded on double-precision numbers.

If the input is very small (i.e. $|b| \leq 2^{-40}$), $tsb$ and $tcb$
are not calculated but directly set to $0$, to avoid any problem with
subnormalized numbers.


At this stage, we have computed all the needed sub-terms before the
final reconstruction, which is done in two steps, corresponding to the
two-step range-reduction. The reconstruction is computed in
double-double arithmetic.  In the first reconstruction, some sub-terms
can be ignored without any loss of precision, due to their very small
relative values.  For this step, it exists a particular case, when
$index = 0$, since it is the only case where $|\sinh(a)| < 2^{-9}$
($\sinh(a) = 0$).  Now we have the definitive values of $\cosh(y)$ and
$\sinh(y)$.

In the second reconstruction, we begin by computing all needed
products before adding their results (i.e. $2^{k-1}\cosh(y)$,
$2^{k-1}\sinh(y)$,...).  Computations are also done using double
double arithmetics, with the Add22 function.



\subsection{Error analysis}

Many of the previous computations can introduce some error.
\begin{itemize}
\item{First range reduction}
We have to consider two different cases:
\begin{itemize}
\item{$|x| \leq \frac{ln(2)}{2}$}

We have $k = 0$, and there is no reduction, and no term of error.
\item{$|x| > \frac{ln(2)}{2}$}

  We have $k \neq 0$, and we must compute the term of error introduced
  by the range reduction.  Since $k$ is an integer, we can assume that
  there is no error on it.  $ln(2)$ is a constant which is stored in
  the function in double-double, and we have $ln(2) =
  ln2_{hi} + ln2_{lo} + \maxabserr{repr\_ln2}$, where
  $|\maxabserr{repr ln2}| \leq 1.94e-31$.  The total absolute error of
  this reduction is $\maxabserr{range\_reduc} = 3.437e-27$, so the
  maximum relative error is $\maxrelerr{range\_reduc} = 9.9e-27$ (we
  have $|x| \geq 0.36$), and that represents about $86.38$ bits of
  precision.

\end{itemize}

\item{Second range reduction} This range reduction is exact (we only
  cut y in two parts, with no multiplication nor division), so no new
  term of error is introduced.

\item{Error in tabulation} Since $\cosh(a)$ and $\sinh(a)$ are stored as
  double-doubles, and since they are transcendental numbers (when $a
  \neq 0$), some error is done on their approximation.  A simple Maple
  procedure can compute this error, which is about $\maxabserr{ca} =
  6.08e-33$ for cosh and $\maxabserr{sa} = 1.47e-33$ for sinh. That is
  large overkill compared to precision on other values.

\item{Error in polynomial approximations} We use the
  $errlist\_quickphase$ and $compute\_horner\_rounding\_error$ Maple
  procedures to compute thes errors on $tcb$ and $tsb$, which are
  $\maxabserr{rounding\_cosh} = 6.35e-22$ and
  $\maxabserr{rounding\_sinh} = 1.94e-22$. Then there is the approximation error. The sum of theses errors
  gives $\maxabserr{tcb} = 6.35e-22$ and $\maxabserr{tsb} = 1.11e-21$.

\item{First reconstruction} This reconstruction is done by adding all
  the pre-calculated terms ($tcb$, $tsb$, $ca = \cosh(a)$, $sa =
  \sinh(a)$), in an order which try to minimize the total
  error.$\maxabserr{sh} = 2.10e-25$. Maple scripts are used to compute
  the error, since there are many terms.  There are 2 different cases:
\begin{itemize}

\item{$a = 0$}

  $ch_{hi}+ch_{lo} = \widehat{\cosh(\widehat{y})} + \abserr{cosh0}$, where $|\abserr{cosh0}| \leq \maxabserr{cosh0} = 6.35e-22$, and 
  $\sinh(y) = \widehat{\sinh(\widehat{y})} + \abserr{sinh0}$, where $|\abserr{sinh0}| \leq  \maxabserr{sinh0} = 5.4e-20$.

\item{$a \neq 0$}

  $ch_{hi}+ch_{lo} = \widehat{\cosh(\widehat{y})} + \abserr{cosh1}$, where $|\abserr{cosh1}| \leq \maxabserr{cosh1} = 2.39e-20$, and 
  $\sinh(y) = \widehat{\sinh(\widehat{y})} + \abserr{sinh1}$, where $|\abserr{sinh1}| \leq  \maxabserr{sinh1} = 1.09e-22$.
\end{itemize}

\item{Second reconstruction} This reconstruction is based on
  multiplying the obtained results before adding them. The products
  are exact since each product has a factor which a power of 2.  We
  have to leave absolute errors for relative errors, since the range
  of values returned by $\cosh$ is too large.  We will distinguish three
  different cases:
\begin{itemize}
\item{$|k| \leq 35$}
  All terms must be computed. We have $ch_{hi} + ch_{lo} = \widehat{\cosh(\widehat{x})}(1+\relerr{ch})$, where $|\relerr{ch}| \leq \maxrelerr{ch} = 7.66e-19$
\item{$k > 35$}
  In this case, the terms corresponding to $e^{-x}$  are neglected, with an error smaller than $2^{-68}$. We have $ch_{hi} + ch_{lo} = \widehat{\cosh(\widehat{x})}(1+\relerr{ch})$, where $|\relerr{ch}| \leq \maxrelerr{ch} = 7.69e-19$
\item{$k < -35$}
  This case is symmetric to the previous one, we just have to remplace k by -k.
\end{itemize}
\end{itemize} 


\newpage
\subsection{Details of computer program}

The procedures \texttt{cosh\_quick} and \texttt{sinh\_quick} contain the computation respectively shared by the 
functions \texttt{cosh\_rn}, \texttt{cosh\_ru}, \texttt{cosh\_rd} and \texttt{cosh\_rz} in one hand, and by the 
functions \texttt{sinh\_rn}, \texttt{sinh\_ru}, \texttt{sinh\_rd} and \texttt{sinh\_rz} in the other hand.
The eight functions \texttt{cosh\_rX} and \texttt{sinh\_rX} call \texttt{cosh\_quick} or \texttt{sinh\_quick} with an integer which represent the choosen rounding mode.
We will begin to prove the cosh function, and then we will prove the sinh function. Since both functions share a lot a code, only the different part between cosh and sinh will be proven for the sinh. 

\subsubsection{Exceptional cases and argument reduction}

This part  is shown for \texttt{cosh\_rn}, but it is quite identical for the three other functions.

\begin{lstlisting}[caption={Exceptional cases},firstnumber=1]

 double cosh_rn(double x){ 
  db_number y;
  y.d = x;
  y.i[HI] = y.i[HI] & 0x7FFFFFFF;     /* to get the absolute value of the input */
  if (y.d > max_input_ch.d) { /* out of range */
    y.i[LO] = 0; y.i[HI] = 0x7FF00000; return (y.d);
  }
  if ((y.i[HI] & 0x7FF00000) >= (0x7FF00000)) {    /*particular cases : QNaN, SNaN, +- oo*/
   return (y.d);
  }
  return(cosh_quick(x, RN));
}
\end{lstlisting}

\begin{tabular}{ll}
Lines  3 &  Initialize y\\
Line 4 & Get the absolute value of y by removing the first bit.\\
Line 5 & Test if $\cosh(|x|) = \cosh(x)$ is representable as a double.\\
Line 6 & If this test is true, we must return $\infty$.\\
Line 8 & Test if $|x|$ is a special case, like NaN or $\infty$\\
Line 9 & If this test is true, we must return $|x|$ \\
Line 11 & $x$ is a correct input, we can return cosh\_quick. \\
\end{tabular}



\subsubsection{Procedure cosh\_quick}

\begin{lstlisting}[caption={Procedure \texttt{cosh\_quick} - variables},firstnumber=1]

 double cosh_quick(double x, int rounding_mode){

  /*some variable declarations */
  int k;
  db_number y;
  double res_hi, res_lo;
  double ch_hi, ch_lo, sh_hi, sh_lo;/* cosh(x) = (ch_hi + ch_lo)*(cosh(k*ln(2)) + (sh_hi + sh_lo)*(sinh(k*ln(2))) */
  db_number  table_index_float;
  int table_index;
  double temp_hi, temp_lo, temp;/* some temporary variables */
  double b_hi, b_lo,b_ca_hi, b_ca_lo, b_sa_hi, b_sa_lo;
  double ca_hi, ca_lo, sa_hi, sa_lo; /*will be the tabulated values */
  double tcb_hi, tsb_hi; /*results of polynomial approximations*/
  double square_y_hi;
  double ch_2_pk_hi, ch_2_pk_lo, ch_2_mk_hi, ch_2_mk_lo;
  double sh_2_pk_hi, sh_2_pk_lo, sh_2_mk_hi, sh_2_mk_lo;
  db_number two_p_plus_k, two_p_minus_k; /* 2^(k-1) + 2^(-k-1) */
  db_number absyh, absyl, u53, u;

\end{lstlisting}

Here there are all the variables which will be used in the code.

\subsubsection{First range reduction}

\begin{lstlisting}[caption={Procedure \texttt{cosh\_quick} - first range reduction},firstnumber=19]

  /* Now we can do the first range reduction*/
  DOUBLE2INT(k, x * inv_ln_2.d)
    if (k != 0){ /* b_hi+b_lo =  x - (ln2_hi + ln2_lo) * k */
      temp_hi = x - ln2_hi.d * k;                                         
      temp_lo = -ln2_lo.d * k;                                          
      Add12Cond(b_hi, b_lo, temp_hi, temp_lo); 
    }
    else {                                                         
      b_hi = x;  b_lo = 0.;
    }                                                               
\end{lstlisting}

\begin{tabular}{ll}
Line 20 & Put in k the closest integer of x * inv\_ln\_2. \\
        & We use the property of DOUBLE2INT that convert a floating-point number in rouding to nearest mode. \\
        & By its definition, $k$ satisfies the following properties:\\
        &  $\lfloor x \times inv\_ln2 \rfloor \leq k \leq \lceil x \times inv\_ln2\rceil$ \\
        & $|k| \leq \frac{x}{2} \times inv\_ln2$ \\
        & since $|x| \leq 710.475...$, we have $|k| \leq 1025$, so $k$ is coded on at most 11 bits. \\
Line 21 & First case : $k \neq 0$ \\
        & We have by contruction : $ln2_{hi} + ln2_{lo} = \ln(2) + \abserr{repr\_ln2}$, where $|\abserr{repr\_ln2}| \leq \maxabserr{repr\_ln2} = 1.95e-31$.\\
        & the last 11 bits of $ln_{hi}$ are set to zero by its construction \\
Line 22 & the $ln2_{hi}  k$ product is exact since $k$ is coded on at most 11 bits and the last 11 bits of $ln2_{hi}$ are zeros \\
        & we have to use the properties verified by $k$: $x  inv\_ln2 - 1 \leq k \leq x  inv\_ln2 + 1$\\
        & if $x \geq 0$ \\
        & we have $ k \geq 1$ and then $x \geq \frac{\ln(2)}{2}$, so $(x  inv\_ln2 + 1)ln2_{hi} \leq 2 x$\\
        & since $|k| \leq \frac{x}{2} \times inv\_ln2$, we have $\frac{x}{2} \leq (x  inv\_ln2 - 1)ln2_{hi}$\\
        & and then we have $\frac{x}{2} \leq k  ln2_{hi} \leq 2 x$\\
        & we can apply the Sterbenz theorem to prove that the result of this line is exact\\
        & if $x \leq 0$\\
        & we can use the same reasoning and then apply the Sterbenz theorem\\
        & and this line of code is always exact. \\
Line 23 & this product is not exact, we can loose at most 11 bits of precision\\
        & there is an error of $\abserr{round}$ which satisfies $|\abserr{round}|\leq \maxabserr{round} = 3.15e-30$ on $ln2_{lo}$ \\
        & so a bound to the maximal absolute error is $k_{max} \maxabserr{round}$\\
Line 24 & We do an Add12 to have well-aligned double doubles in $b_{hi}$ and $b_{lo}$\\
        & The conditionnal version is used since temp\_hi  can be zero if $x$ is very close to $k ln(2)$.\\
        & The total absolute error is bounded by $\maxabserr{b} = 3.43e-27$ \\
Line 27 & We have $k = 0$. We needn't to do any reduction, so $b_{hi} + b_{lo} = x$ exactly.\\
\end{tabular}

At this stage, we have $b_{hi} + b_{lo} = \widehat{y} + \abserr{b}$, where $|\abserr{b}| \leq \maxabserr{b} = 3.43e-24$. Now we will write $y = a + b$, where $a = 2^{-8}  index$. 

\subsubsection{Second range reduction}

\begin{lstlisting}[caption={Procedure \texttt{cosh\_quick} - second range reduction},firstnumber=29]

  /*we'll construct 2 constants for the last reconstruction */
  two_p_plus_k.i[LO] = 0;
  two_p_plus_k.i[HI] = (k-1+1023) << 20;
  two_p_minus_k.i[LO] = 0;
  two_p_minus_k.i[HI] = (-k-1+1023) << 20;

  /* at this stage, we've done the first range reduction : we have b_hi + b_lo  between -ln(2)/2 and ln(2)/2 */
  /* now we can do the second range reduction */
  /* we'll get the 8 leading bits of b_hi */
  table_index_float.d = b_hi + two_43_44.d;
  /*this add do the float equivalent of a rotation to the right, since -0.5 <= b_hi <= 0.5*/
  table_index = LO(table_index_float.d);/* -89 <= table_index <= 89 */
  table_index_float.d -= two_43_44.d;
  table_index += bias; /* to have only positive values */
  b_hi -= table_index_float.d;/* to remove the 8 leading bits*/
  /* since b_hi was between -2^-1 and 2^1, we now have b_hi between -2^-9 and 2^-9 */  
\end{lstlisting}

\begin{tabular}{ll}
Line 30-33 & Put in \texttt{two\_p\_plus\_k} and \texttt{two\_p\_minus\_k} the exact values of $2^{k-1}$ and $2^{-k-1}$.\\
Line 38-44 & The goal of the second range reduction is to write $y$ as $y = index  2^{-8} + b$ \\
           & We have $|y| \leq \frac{ln(2)}{2} \leq \frac{1}{2}$ \\
           & so $2^{44} \leq 2^{44} + 2^{43} + y \leq 2^{44} + 2^{43} + 2^{42}$ \\
           & since the mantissa counts 53 bits, only the part above $2^{-8}$ si kept in table\_index\_float\\
           & It is easy to show that we have $-89 \leq table\_index \leq 89$ \\
           & so we can add $bias = 89$ to $table\_index$ to have only positive values. \\
           & then we remove this bits of $y$ to obtain the final $b = b_{hi} + b_{lo}$ \\
           & all these operations are exact, so the final absolute error doesn't increase \\

\end{tabular}


\subsubsection{Polynomial evaluation - First reconstruction}

\begin{lstlisting}[caption={Procedure \texttt{cosh\_quick} - polynomial evaluation - first reconstruction},firstnumber=45]
  y.d = b_hi;
  /*   first, y�  */
  square_b_hi = b_hi * b_hi;
  /* effective computation of the polynomial approximation */
  
  if (((y.i[HI])&(0x7FFFFFFF)) < (two_minus_30.i[HI])) {
    tcb_hi = 0;
    tsb_hi = 0;
  }
  else {
    /*   second, cosh(b) = b� * (1/2 + b� * (1/24 + b� * 1/720)) */
    tcb_hi = (square_b_hi)* (c2.d + square_b_hi * (c4.d + square_b_hi * c6.d));
    tsb_hi = square_b_hi * (s3.d + square_b_hi * (s5.d + square_b_hi * s7.d));
  }
 

  if( table_index != bias) {
    /* we get the tabulated the tabulated values */
    ca_hi = cosh_sinh_table[table_index][0].d;
    ca_lo = cosh_sinh_table[table_index][1].d;
    sa_hi = cosh_sinh_table[table_index][2].d;
    sa_lo = cosh_sinh_table[table_index][3].d;
    
    /* first reconstruction of the cosh (corresponding to the second range reduction) */
    Mul12(&b_sa_hi,&b_sa_lo, sa_hi, b_hi);
    temp =  ((((((ca_lo + (b_hi * sa_lo)) + b_lo * sa_hi) + b_sa_lo) + (b_sa_hi * tsb_hi)) + ca_hi * tcb_hi) + b_sa_hi);
    Add12Cond(ch_hi, ch_lo, ca_hi, temp);
      /* first reconstruction for the sinh (corresponding to the second range reduction) */
  }
  else {
    Add12Cond(ch_hi, ch_lo, (double) 1, tcb_hi);
  }
  
  
\end{lstlisting}

\begin{tabular}{ll}
Line 45 & Put in $y$ the value of $b_{hi}$, so we can use its hexadecimal aspect \\
Line 47    & Put $b^2$ in $square\_b_{hi}$. We have $square\_b_{hi} = \widehat{b} + \abserr{square\_b}$, where $|\abserr{square\_b}| \leq \maxabserr{square\_b} = 4.23e-22$ \\
Line 50    & Match $b_{hi}$ and then $b$ with $2^{-40}$\\
Line 51-52 & If $|b| \leq 2^{-40}$, we will have $|tcb|,[tsb| \leq \maxabserr{square\_b}$, so we can directly set $tcb$ and $tsb$ to zero: \\
           & converning the mathematical values, we have $|\widehat{tcb}|, |\widehat{tsb}| \leq 2^{-24}$. \\
           & We can avoid by this way any problem with subnormalized numbers. \\
Line 55-56 & Polynomial evaluation of $\cosh(x)-1$ and $\frac{\sinh(x)}{x}-1$, following the H�rner scheme. \\
           & A maple procedure is used to compute the error on this computations\\
           & There are 2 reasons for the total error :\\
           & the effective computations, since all operations are done with 53 bits of precision.\\
           & the mathematical approximation, since we use polynoms \\
           & finally, we have $tcb = \widehat{\cosh(\widehat{b}-1)} + \abserr{tcb}$, where $|\abserr{tcb}| \leq \maxabserr{tcb} = 6.35e-22$, \\
           & and $tsb = \widehat{(\frac{\sinh(\widehat{b})}{\widehat{b}}-1)} + \abserr{tsb}$, where $|\abserr{tsb}| \leq \maxabserr{tsb} = 1.11e-21$ \\
Line 60    & If $y$ is very close to $0$, we have the 8 bits of the second range reduction which are null \\
Line 62-65 & We get tabulated values for $\cosh(a)$ and $\sinh(a)$. They are tabulated as double doubles: \\
           & we have $ca_{hi} + ca_{lo} = \widehat{\cosh(\widehat{a})} + \abserr{ca}$, where $|\abserr{ca}| \leq \maxabserr{ca} = 6.08e-33$, \\
           & and $sa_{hi} + sa_{lo} = \widehat{\sinh(\widehat{a})} + \abserr{sa}$, where $|\abserr{sa}| \leq \maxabserr{sa} = 1.47e-33$, \\
Line 68    & $b\_sa_{hi} + b\_sa_{lo} = sa_{hi}  b_{hi}$. This product is exact. \\
Line 69-70 & it is the reconstruction : $\cosh(y) = \cosh(a)(1+tcb) + \sinh(a)b(1+tsb)$ \\
           & A maple procedure is used to compute the error done in this reconstruction. \\
           & We have $ch_{hi}+ch_{lo} = \widehat{\cosh(\widehat{y})} + \abserr{cosh1}$, where $|\abserr{cosh1}| \leq \maxabserr{cosh1} = 2.39e-20$\\
Line 75    & If $y$ is very close to $0$, we have $a = 0$ and $\cosh(y) = \cosh(b) = 1 + tcb$. \\
           & This addition is exact, so no error is introduced. \\
           & We have $ch_{hi}+ch_{lo} = \widehat{\cosh(\widehat{y})} + \abserr{cosh0}$, where $|\abserr{cosh0}| \leq \maxabserr{cosh0} = 6.35e-22$\\
\end{tabular}


\subsubsection{Second reconstruction}

\begin{lstlisting}[caption={Procedure \texttt{cosh\_quick} - reconstruction},firstnumber=77]
  if(k != 0) {
    if( table_index != bias) {
      /* first reconstruction for the sinh (corresponding to the second range reduction) */
      Mul12(&b_ca_hi , &b_ca_lo, ca_hi, b_hi);
      temp = (((((sa_lo + (b_lo * ca_hi)) + (b_hi * ca_lo)) + b_ca_lo) + (sa_hi*tcb_hi)) + (b_ca_hi * tsb_hi));
      Add12(temp_hi, temp_lo, b_ca_hi, temp);
      Add22Cond(&sh_hi, &sh_lo, sa_hi, (double) 0, temp_hi, temp_lo);
    }
    else {
      Add12Cond(sh_hi, sh_lo, b_hi, tsb_hi * b_hi + b_lo);
    }
    if((k < 35) && (k > -35) )
      {
        ch_2_pk_hi = ch_hi * two_p_plus_k.d;
        ch_2_pk_lo = ch_lo * two_p_plus_k.d;
        ch_2_mk_hi = ch_hi * two_p_minus_k.d;
        ch_2_mk_lo = ch_lo * two_p_minus_k.d;
        sh_2_pk_hi = sh_hi * two_p_plus_k.d;
        sh_2_pk_lo = sh_lo * two_p_plus_k.d;
        sh_2_mk_hi = -1 * sh_hi * two_p_minus_k.d;
        sh_2_mk_lo = -1 * sh_lo * two_p_minus_k.d;
        
        Add22Cond(&res_hi, &res_lo, ch_2_mk_hi, ch_2_mk_lo, sh_2_mk_hi, sh_2_mk_lo);
        Add22Cond(&ch_2_mk_hi, &ch_2_mk_lo , sh_2_pk_hi, sh_2_pk_lo, res_hi, res_lo);
        Add22Cond(&res_hi, &res_lo, ch_2_pk_hi, ch_2_pk_lo, ch_2_mk_hi, ch_2_mk_lo);
      } 
    else if (k >= 35) 
      {
        ch_2_pk_hi = ch_hi * two_p_plus_k.d;
        ch_2_pk_lo = ch_lo * two_p_plus_k.d;
        sh_2_pk_hi = sh_hi * two_p_plus_k.d;
        sh_2_pk_lo = sh_lo * two_p_plus_k.d;
        Add22Cond(&res_hi, &res_lo, ch_2_pk_hi, ch_2_pk_lo, sh_2_pk_hi, sh_2_pk_lo);
      }
    else /* if (k <= -35) */
      {
        ch_2_mk_hi = ch_hi * two_p_minus_k.d;
        ch_2_mk_lo = ch_lo * two_p_minus_k.d;
        sh_2_mk_hi = -1 * sh_hi * two_p_minus_k.d;
        sh_2_mk_lo = -1 * sh_lo * two_p_minus_k.d;
        Add22Cond(&res_hi, &res_lo, ch_2_mk_hi, ch_2_mk_lo, sh_2_mk_hi, sh_2_mk_lo);
      }
  }
  else {
    res_hi = ch_hi;
    res_lo = ch_lo;
  }

\end{lstlisting}

\begin{tabular}{ll}
Line 77    & Test if $k = 0$ or not \\
Line 78-87 & We have $k \neq 0$, so we must compute $\sinh(y)$ \\
           & This computation is done like the computation of $\cosh(h)$ \\
           & We can use an Add12 (instead of Add12Cond) since $b_{hi}  ca_{hi} \geq  temp$ \\
           & A maple script gives $\sinh(y) = \widehat{\sinh(\widehat{y})} + \abserr{sinh1}$, where $|\abserr{sinh1}| \leq  \maxabserr{sinh1} = 1.09e-22$ \\

           & and $|\abserr{sinh1}| \leq  \maxabserr{sinh0} = 5.4e-20$ (when $\sinh(a) = 0$) \\
Line 89    & We have $k\neq 0$, and $|k| \leq 35$ \\
Line 91-98 & we multiply $\sinh(y)$ and $\cosh(y)$ by powers of 2, so these products are exact \\
Line 100-102 & A maple script is used to compute the error: \\
             & We have $ch_{hi} + ch_{lo} = \widehat{\cosh(\widehat{x})}(1+\relerr{ch})$, where $|\relerr{ch}| \leq \maxrelerr{ch} = 7.66e-19$ \\
Line 104   & $k \geq 35$ \\
Line 106-109 & we multiply $\sinh(y)$ and $\cosh(y)$ by powers of 2, so these products are exact \\
           & Some terms are not computed, since they are too little \\
Line 110   &  A maple script is used to compute the error: \\
             & We have $ch_{hi} + ch_{lo} = \widehat{\cosh(\widehat{x})}(1+\relerr{ch})$, where $|\relerr{ch}| \leq \maxrelerr{ch} = 7.69e-19$ \\
Line 112   & $k \leq -35$ \\
Line 114-118 & this case is symmetric to the previous one. \\
           & We also have $ch_{hi} + ch_{lo} = \widehat{\cosh(\widehat{x})}(1+\relerr{ch})$, where $|\relerr{ch}| \leq \maxrelerr{ch} = 7.69e-19$ \\
Line 121   & we now have $k = 0$ \\
           & Since we have $1 \leq \cosh(x)$, we have  $\maxrelerr{ch} \leq max(\maxabserr{cosh0},\maxabserr{cosh1}) = 2.39e-20$ \\
\end{tabular}

At this, stage, we have $ch_{hi} + ch_{lo} = \widehat{\cosh(\widehat{x})}(1+\relerr{ch})$, where  $|\relerr{ch}| \leq \maxrelerr{ch} = 7.69e-19 = 2^{-60.17}$ .


\subsection{Rounding}

\subsubsection{Rounding to the nearest}

The code for rounding is strictly identical to that of
Theorem~\ref{th:roundingRN1}.  The condition to this theorem that
$\mathtt{res\_hi}\ge 2^{-1022+53}$ is ensured by the image domain of
the $\cosh$, since $\cosh(x) \geq 1$. The rounding constant

\subsection{Directed rounding}
Here again, the code is strictly identical to that of
Theorem~\ref{th:roundingDirected}, and the conditions to this theorem
are ensured by the image domain of the cosh.




%%%%%%%%%%%%%%%%%%%%%%%%%%%%%%%%%%%%%%%%%%%%%%%%%%%%%%%%%%%%%%%%%%%%%%%%%%%%%%%%%%%%%
\section{Accurate phase}

An SCS-based exponential function already exists in \texttt{crlibm},
accurate to $2^{-158}$, a large overkill for computing the yperbolic
functions. We therefore use the equation

 $$\cosh(x) = \frac{e^x +  e^{-x}}{2}\quad.$$

 It is still possible to optimise out the $e^{-x}$ for large
 arguments. Indeed, for $|k|>65$ we have $e^{-x}<2^{-128}e^x$ so the
 relative error due to neglecting $e^{-x}$ is much smaller than the
 worst case accuracy required to decide rounding for such large values
 (in the order of $2^{-115}$ for both functions). This is reflected in the following code:


\begin{lstlisting}[caption={Procedure \texttt{cosh\_accurate}},firstnumber=177]
  if ((k > -65) && (k < 65)) {
    exp_SC(exp_scs, x);
    scs_inv(exp_minus_scs, exp_scs);
    scs_add(res_scs, exp_scs, exp_minus_scs);
    scs_div_2(res_scs);
  }
  else if (k >= 65) {
    exp_SC(res_scs, x);
    scs_div_2(res_scs);
  }
  else {
    exp_SC(res_scs, -x);
    scs_div_2(res_scs);
  }
\end{lstlisting}

\begin{tabular}{ll}
Line 177   & We will distinguish three cases, since if $|x|$ is large enough, we can ignore some computations. \\
Line 178-181 & The formula used  is $\cosh(x) =  (exp(x) + \frac{1}{exp(x)})/2$. \\
           & The final result has at least 150 bits of accuracy,\\
           & and that is enough for a correct rounding \\
Line 182   & If $|k| \geq 65$, $e^{-x}$ is negligible, since we have $e^{-x} \leq 2^{-128} e^x$. \\
Line 186   & This case is symetric to the previous one. \\
\end{tabular}

%%%%%%%%%%%%%%%%%%%%%%%%%%%%%%%%%%%%%%%%%%%%%%%%%%%%%%%%%%%%%
\section{Analysis of cosh performance}
\label{section:cosh_results}
The input numbers for the performance tests given here are random
positive double-precision numbers with a normal distribution on the
exponents. More precisely, we take random 63-bit integers and cast
them into double-precision numbers.  


In average, the second step is taken in 0.13\% of the calls.


\subsection{Speed}
Table \ref{tbl:cosh_abstime} (produced by the \texttt{crlibm\_testperf}
executable) gives absolute timings for a variety of processors and
operating systems.  

\begin{table}[!htb]
\begin{center}
\renewcommand{\arraystretch}{1.2}
\begin{tabular}{|l|r|r|r|}
\hline
 \multicolumn{4}{|c|}{Processor / system / compiler}   \\ 
 \hline
                         & min time      & max time      & avg time \\ 
 \hline
 \texttt{libm}           & 116          & 920           &        323 \\ 
 \hline
  \texttt{mpfr}          & 19464        & 224896        &      30338 \\ 
 \hline
 \texttt{crlibm}         & 132          & 55036         &        547 \\ 
 \hline

 \hline
  \multicolumn{4}{|c|}{PowerPC G4 / MacOS X / gcc2.95}   \\
 \hline
                         & min time      & max time      & avg time \\
 \hline
 \texttt{libm}           & 15           & 23            &         22 \\
 \hline
  \texttt{mpfr}          & 1315         & 11021         &       1707 \\
 \hline
 \texttt{crlibm}         & 6            & 1195          &         24 \\
 \hline
\end{tabular}
\end{center}
\caption{Absolute timings for the hyperbolic cosine
  \label{tbl:cosh_abstime}}
\end{table}

Contributions to this table for new processors/OS/compiler combinations are welcome.



